% Lines starting with a percent sign (%) are comments. LaTeX will
% not process those lines. Similarly, everything after a percent
% sign in a line is considered a comment. To produce a percent sign
% in the output, write \% (backslash followed by the percent sign).
% ==================================================================
% Usage instructions:
% ------------------------------------------------------------------ 
% The file is heavily commented so that you know what the various
% commands do. Feel free to remove any comments you don't need from
% your own copy. When redistributing the example thesis file, please
% retain all the comments for the benefit of other thesis writers!
% ==================================================================
% Compilation instructions:
% ------------------------------------------------------------------
% Use pdflatex to compile! Input images are expected as PDF files.
% Example compilation:
% ------------------------------------------------------------------
% > pdflatex thesis-example.tex
% > bibtex thesis-example
% > pdflatex thesis-example.tex
% > pdflatex thesis-example.tex
% ------------------------------------------------------------------
% You need to run pdflatex multiple times so that all the cross-references
% are fixed. pdflatex will tell you if you need to re-run it (a warning
% will be issued)
% ------------------------------------------------------------------
% Compilation has been tested to work in ukk.cs.hut.fi and kosh.hut.fi
% - if you have problems of missing .sty -files, then the local LaTeX
% environment does not have all the required packages installed.
% For example, when compiling in vipunen.hut.fi, you get an error that
% tikz.sty is missing - in this case you must either compile somewhere
% else, or you cannot use TikZ graphics in your thesis and must therefore
% remove or comment out the tikz package and all the tikz definitions.
% ------------------------------------------------------------------

% General information
% ==================================================================
% Package documentation:
%
% The comments often refer to package documentation. (Almost) all LaTeX
% packages have documentation accompanying them, so you can read the
% package documentation for further information. When a package 'xxx' is
% installed to your local LaTeX environment (the document compiles
% when you have \usepackage{xxx} and LaTeX does not complain), you can
% find the documentation somewhere in the local LaTeX texmf directory
% hierarchy. In ukk.cs.hut.fi, this is /usr/texlive/2008/texmf-dist,
% and the documentation for the titlesec package (for example) can be
% found at /usr/texlive/2008/texmf-dist/doc/latex/titlesec/titlesec.pdf.
% Most often the documentation is located as a PDF file in
% /usr/texlive/2008/texmf-dist/doc/latex/xxx, where xxx is the package name;
% however, documentation for TikZ is in
% /usr/texlive/2008/texmf-dist/doc/latex/generic/pgf/pgfmanual.pdf
% (this is because TikZ is a front-end for PGF, which is meant to be a
% generic portable graphics format for LaTeX).
% You can try to look for the package manual using the ``find'' shell
% command in Linux machines; the find databases are up-to-date at least
% in ukk.cs.hut.fi. Just type ``find xxx'', where xxx is the package
% name, and you should find a documentation file.
% Note that in some packages, the documentation is in the DVI file
% format. In this case, you can copy the DVI file to your home directory,
% and convert it to PDF with the dvipdfm command (or you can read the
% DVI file directly with a DVI viewer).
%
% If you can't find the documentation for a package, just try Googling
% for ``latex packagename''; most often you can get a direct link to the
% package manual in PDF format.
% ------------------------------------------------------------------


% Document class for the thesis is report
% ------------------------------------------------------------------
% You can change this but do so at your own risk - it may break other things.
% Note that the option pdftext is used for pdflatex; there is no
% pdflatex option.
% ------------------------------------------------------------------
\documentclass[12pt,a4paper,oneside,pdftex]{report}

% The input files (tex files) are encoded with the latin-1 encoding
% (ISO-8859-1 works). Change the latin1-option if you use UTF8
% (at some point LaTeX did not work with UTF8, but I'm not sure
% what the current situation is)
\usepackage[utf8]{inputenc}
% OT1 font encoding seems to work better than T1. Check the rendered
% PDF file to see if the fonts are encoded properly as vectors (instead
% of rendered bitmaps). You can do this by zooming very close to any letter
% - if the letter is shown pixelated, you should change this setting
% (try commenting out the entire line, for example).
\usepackage[OT1]{fontenc}
% The babel package provides hyphenating instructions for LaTeX. Give
% the languages you wish to use in your thesis as options to the babel
% package (as shown below). You can remove any language you are not
% going to use.
% Examples of valid language codes: english (or USenglish), british,
% finnish, swedish; and so on.
\usepackage[finnish,swedish,english]{babel}

\usepackage{enumerate}

% Font selection
% ------------------------------------------------------------------
% The default LaTeX font is a very good font for rendering your
% thesis. It is a very professional font, which will always be
% accepted.
% If you, however, wish to spicen up your thesis, you can try out
% these font variants by uncommenting one of the following lines
% (or by finding another font package). The fonts shown here are
% all fonts that you could use in your thesis (not too silly).
% Changing the font causes the layouts to shift a bit; you many
% need to manually adjust some layouts. Check the warning messages
% LaTeX gives you.
% ------------------------------------------------------------------
% To find another font, check out the font catalogue from
% http://www.tug.dk/FontCatalogue/mathfonts.html
% This link points to the list of fonts that support maths, but
% that's a fairly important point for master's theses.
% ------------------------------------------------------------------
% <rant>
% Remember, there is no excuse to use Comic Sans, ever, in any
% situation! (Well, maybe in speech bubbles in comics, but there
% are better options for those too)
% </rant>

% \usepackage{palatino}
% \usepackage{tgpagella}



% Optional packages
% ------------------------------------------------------------------
% Select those packages that you need for your thesis. You may delete
% or comment the rest.

% Natbib allows you to select the format of the bibliography references.
% The first example uses numbered citations:
%\usepackage[square,sort&compress,numbers]{natbib}
% The second example uses author-year citations.
% If you use author-year citations, change the bibliography style (below);
% acm style does not work with author-year citations.
% Also, you should use \citet (cite in text) when you wish to refer
% to the author directly (\citet{blaablaa} said blaa blaa), and
% \citep when you wish to refer similarly than with numbered citations
% (It has been said that blaa blaa~\citep{blaablaa}).
%\usepackage[square]{natbib}

% The alltt package provides an all-teletype environment that acts
% like verbatim but you can use LaTeX commands in it. Uncomment if
% you want to use this environment.
% \usepackage{alltt}

% The eurosym package provides a euro symbol. Use with \euro{}
\usepackage{eurosym}

% Verbatim provides a standard teletype environment that renderes
% the text exactly as written in the tex file. Useful for code
% snippets (although you can also use the listings package to get
% automatic code formatting).
\usepackage{verbatim}
\usepackage{float}

% The listing package provides automatic code formatting utilities
% so that you can copy-paste code examples and have them rendered
% nicely. See the package documentation for details.
% \usepackage{listings}

% The fancuvrb package provides fancier verbatim environments
% (you can, for example, put borders around the verbatim text area
% and so on). See package for details.
% \usepackage{fancyvrb}

% Supertabular provides a tabular environment that can span multiple
% pages.
%\usepackage{supertabular}
% Longtable provides a tabular environment that can span multiple
% pages. This is used in the example acronyms file.
\usepackage{longtable}

% The fancyhdr package allows you to set your the page headers
% manually, and allows you to add separator lines and so on.
% Check the package documentation.
% \usepackage{fancyhdr}

% Subfigure package allows you to use subfigures (i.e. many subfigures
% within one figure environment). These can have different labels and
% they are numbered automatically. Check the package documentation.
\usepackage{subfigure}

% The titlesec package can be used to alter the look of the titles
% of sections, chapters, and so on. This example uses the ``medium''
% package option which sets the titles to a medium size, making them
% a bit smaller than what is the default. You can fine-tune the
% title fonts and sizes by using the package options. See the package
% documentation.
\usepackage[medium]{titlesec}

% The TikZ package allows you to create professional technical figures.
% The learning curve is quite steep, but it is definitely worth it if
% you wish to have really good-looking technical figures.
\usepackage{tikz}
% You also need to specify which TikZ libraries you use
\usetikzlibrary{positioning}
\usetikzlibrary{calc}
\usetikzlibrary{arrows}
\usetikzlibrary{decorations.pathmorphing,decorations.markings}
\usetikzlibrary{shapes}
\usetikzlibrary{patterns}


% The aalto-thesis package provides typesetting instructions for the
% standard master's thesis parts (abstracts, front page, and so on)
% Load this package second-to-last, just before the hyperref package.
% Options that you can use:
%   mydraft - renders the thesis in draft mode.
%             Do not use for the final version.
%   doublenumbering - [optional] number the first pages of the thesis
%                     with roman numerals (i, ii, iii, ...); and start
%                     arabic numbering (1, 2, 3, ...) only on the
%                     first page of the first chapter
%   twoinstructors  - changes the title of instructors to plural form
%   twosupervisors  - changes the title of supervisors to plural form
%\usepackage[twosupervisors]{aalto-thesis}
%\usepackage[mydraft,doublenumbering]{aalto-thesis}
\usepackage[mydraft]{aalto-thesis}
\usepackage{pdfpages}

% Hyperref
% ------------------------------------------------------------------
% Hyperref creates links from URLs, for references, and creates a
% TOC in the PDF file.
% This package must be the last one you include, because it has
% compatibility issues with many other packages and it fixes
% those issues when it is loaded.
\RequirePackage[pdftex]{hyperref}
% Setup hyperref so that links are clickable but do not look
% different
\hypersetup{colorlinks=false,raiselinks=false,breaklinks=true}
\hypersetup{pdfborder={0 0 0}}
\hypersetup{bookmarksnumbered=true}
% The following line suggests the PDF reader that it should show the
% first level of bookmarks opened in the hierarchical bookmark view.
\hypersetup{bookmarksopen=true,bookmarksopenlevel=1}
% Hyperref can also set up the PDF metadata fields. These are
% set a bit later on, after the thesis setup.


% Thesis setup
% ==================================================================
% Change these to fit your own thesis.
% \COMMAND always refers to the English version;
% \FCOMMAND refers to the Finnish version; and
% \SCOMMAND refers to the Swedish version.
% You may comment/remove those language variants that you do not use
% (but then you must not include the abstracts for that language)
% ------------------------------------------------------------------
% If you do not find the command for a text that is shown in the cover page or
% in the abstract texts, check the aalto-thesis.sty file and locate the text
% from there.
% All the texts are configured in language-specific blocks (lots of commands
% that look like this: \renewcommand{\ATCITY}{Espoo}.
% You can just fix the texts there. Just remember to check all the language
% variants you use (they are all there in the same place).
% ------------------------------------------------------------------
\newcommand{\TITLE}{Improving In-House Software Development \vspace{0 mm} Process:}
\newcommand{\SUBTITLE}{A User-Centered Approach}
\newcommand{\FTITLE}{Yrityksen sisäisen ohjelmistokehitysprosessin parantaminen:}
\newcommand{\FSUBTITLE}{Käyttäjäkeskeinen lähestymistapa}
%\newcommand{\STITLE}{Den stora stygga vargen:}
%\newcommand{\SUBTITLE}{Subtitle}
%\newcommand{\FSUBTITLE}{Alaotsikko}
%\newcommand{\SSUBTITLE}{Lilla Vargens universum}
\newcommand{\DATE}{June 5, 2013}
\newcommand{\FDATE}{5. kesäkuuta 2013}
%\newcommand{\SDATE}{Den 18 Juni 2011}

% Supervisors and instructors
% ------------------------------------------------------------------
% If you have two supervisors, write both names here, separate them with a
% double-backslash (see below for an example)
% Also remember to add the package option ``twosupervisors'' or
% ``twoinstructors'' to the aalto-thesis package so that the titles are in
% plural.
% Example of one supervisor:
\newcommand{\SUPERVISOR}{Professor Marko Nieminen}
\newcommand{\FSUPERVISOR}{Professori Marko Nieminen}
%\newcommand{\SSUPERVISOR}{Professor Antti Ylä-Jääski}
% Example of twosupervisors:
%\newcommand{\SUPERVISOR}{Professor Marko Nieminen\\
%  Professor Marko Nieminen}
%\newcommand{\FSUPERVISOR}{Professori Marko Nieminen\\
%  Professori Marko Nieminen}
%\newcommand{\SSUPERVISOR}{Professor Antti Ylä-Jääski\\
%  Professor Pekka Perustieteilijä}

% If you have only one instructor, just write one name here
\newcommand{\INSTRUCTOR}{Jouni Kuusinen M.Sc. (Tech.)}
\newcommand{\FINSTRUCTOR}{Filosofian maisteri Jouni Kuusinen}
%\newcommand{\SINSTRUCTOR}{Diplomingenjör Olli Ohjaaja}
% If you have two instructors, separate them with \\ to create linefeeds
% \newcommand{\INSTRUCTOR}{Olli Ohjaaja M.Sc. (Tech.)\\
%  Elli Opas M.Sc. (Tech)}
%\newcommand{\FINSTRUCTOR}{Diplomi-insinööri Olli Ohjaaja\\
%  Diplomi-insinööri Elli Opas}
%\newcommand{\SINSTRUCTOR}{Diplomingenjör Olli Ohjaaja\\
%  Diplomingenjör Elli Opas}

% If you have two supervisors, it is common to write the schools
% of the supervisors in the cover page. If the following command is defined,
% then the supervisor names shown here are printed in the cover page. Otherwise,
% the supervisor names defined above are used.
%\newcommand{\COVERSUPERVISOR}{Professor Antti Ylä-Jääski, Aalto University\\
%  Professor Pekka Perustieteilijä, University of Helsinki}

% The same option is for the instructors, if you have multiple instructors.
% \newcommand{\COVERINSTRUCTOR}{Olli Ohjaaja M.Sc. (Tech.), Aalto University\\
%  Elli Opas M.Sc. (Tech), Aalto SCI}


% Other stuff
% ------------------------------------------------------------------
\newcommand{\PROFESSORSHIP}{Usability and User Interfaces}
\newcommand{\FPROFESSORSHIP}{Käytettävyys ja käyttöliittymät}
%\newcommand{\SPROFESSORSHIP}{Datakommunikationsprogram}
% Professorship code is the same in all languages
\newcommand{\PROFCODE}{T-121}
\newcommand{\KEYWORDS}{Usability, ERP, Software development process, Process measurement, Cognitive Walkthrough, Remote usability evaluation, Automatic usability logging, SUS, Contextual Inquiry, ISI, Software process improvement}
\newcommand{\FKEYWORDS}{Käytettävyys, ERP, Ohjelmistokehitysprosessi, Prosessimittaus, Kognitiivinen läpikäynti, Käytettävyyden etäarviointi, Automaattinen käytettävyyslokitus, SUS, Kontekstuaalinen tutkimus, ISI, Ohjelmistoprosessin kehitys}
%\newcommand{\SKEYWORDS}{omsättning, kassaflöde, värdepappersmarknadslagen,
%yrkesutövare, intresseföretag, verifieringskedja}
\newcommand{\LANGUAGE}{English}
\newcommand{\FLANGUAGE}{Englanti}
%\newcommand{\SLANGUAGE}{Engelska}

% Author is the same for all languages
\newcommand{\AUTHOR}{Antti Paananen}


% Currently the English versions are used for the PDF file metadata
% Set the PDF title
\hypersetup{pdftitle={\TITLE\ \SUBTITLE}}
% Set the PDF author
\hypersetup{pdfauthor={\AUTHOR}}
% Set the PDF keywords
\hypersetup{pdfkeywords={\KEYWORDS}}
% Set the PDF subject
\hypersetup{pdfsubject={Master's Thesis}}


% Layout settings
% ------------------------------------------------------------------

% When you write in English, you should use the standard LaTeX
% paragraph formatting: paragraphs are indented, and there is no
% space between paragraphs.
% When writing in Finnish, we often use no indentation in the
% beginning of the paragraph, and there is some space between the
% paragraphs.

% If you write your thesis Finnish, uncomment these lines; if
% you write in English, leave these lines commented!
% \setlength{\parindent}{0pt}
% \setlength{\parskip}{1ex}

% Use this to control how much space there is between each line of text.
% 1 is normal (no extra space), 1.3 is about one-half more space, and
% 1.6 is about double line spacing.
% \linespread{1} % This is the default
% \linespread{1.3}

% Bibliography style
% acm style gives you a basic reference style. It works only with numbered
% references.
\bibliographystyle{acm}
% Plainnat is a plain style that works with both numbered and name citations.
%\bibliographystyle{plainnat}


% Extra hyphenation settings
% ------------------------------------------------------------------
% You can list here all the files that are not hyphenated correctly.
% You can provide many \hyphenation commands and/or separate each word
% with a space inside a single command. Put hyphens in the places where
% a word can be hyphenated.
% Note that (by default) LaTeX will not hyphenate words that already
% have a hyphen in them (for example, if you write ``structure-modification
% operation'', the word structure-modification will never be hyphenated).
% You need a special package to hyphenate those words.
\hyphenation{di-gi-taa-li-sta yksi-suun-tai-sta ana-ly-sis pro-duct}



% The preamble ends here, and the document begins.
% Place all formatting commands and such before this line.
% ------------------------------------------------------------------
\begin{document}
% This command adds a PDF bookmark to the cover page. You may leave
% it out if you don't like it...
\pdfbookmark[0]{Cover page}{bookmark.0.cover}
% This command is defined in aalto-thesis.sty. It controls the page
% numbering based on whether the doublenumbering option is specified
\startcoverpage

% Cover page
% ------------------------------------------------------------------
% Options: finnish, english, and swedish
% These control in which language the cover-page information is shown
\coverpage{english}


% Abstracts
% ------------------------------------------------------------------
% Include an abstract in the language that the thesis is written in,
% and if your native language is Finnish or Swedish, one in that language.

% Abstract in English
% ------------------------------------------------------------------
\thesisabstract{english}{
-
%\fixme{Abstract text goes here (and this is an example how to use fixme).}
%Fixme is a command that helps you identify parts of your thesis that still
%require some work. When compiled in the custom \texttt{mydraft} mode, text
%parts tagged with fixmes are shown in bold and with fixme tags around them. When
%compiled in normal mode, the fixme-tagged text is shown normally (without
%special formatting). The draft mode also causes the ``Draft'' text to appear on
%the front page, alongside with the document compilation date. The custom
%\texttt{mydraft} mode is selected by the \texttt{mydraft} option given for the
%package \texttt{aalto-thesis}, near the top of the \texttt{thesis-example.tex}
%file.
%The thesis example file (\texttt{thesis-example.tex}), all the chapter content
%files (\texttt{1introduction.tex} and so on), and the Aalto style file
%(\texttt{aalto-thesis.sty}) are commented with explanations on how the Aalto
%thesis works. The files also contain some examples on how to customize various
%details of the thesis layout, and of course the example text works as an
%example in itself. Please read the comments and the example text; that should
%get you well on your way!
}

% Abstract in Finnish
% ------------------------------------------------------------------
\thesisabstract{finnish}{
-
}

% Abstract in Swedish
% ------------------------------------------------------------------
%\thesisabstract{swedish}{
%Lilla Vargens universum är det tredje fiktiva universumet inom huvudfäran av de
%tecknade disneyserierna - de övriga två är Kalle Ankas och Musse Piggs
%universum. Figurerna runt Lilla Vargen kommer huvudsakligen frän tre källor ---
%dels persongalleriet i kortfilmen Tre smä grisar frän 1933 och dess uppföljare,
%dels längfilmen Sängen om Södern frän 1946, och dels frän episoden ``Bongo'' i
%längfilmen Pank och fägelfri frän 1947. Framför allt de två första har
%sedermera även kommit att leva vidare, utvidgas och införlivas i varandra genom
%tecknade serier, främst sädana producerade av Western Publishing för
%amerikanska Disneytidningar under ären 1945--1984.

%Världen runt Lilla Vargen är, i jämförelse med den runt Kalle Anka eller Musse
%Pigg, inte helt enhetlig, vilket bland annat märks i Bror Björns skiftande
%personlighet. Den har även varit betydligt mer öppen för influenser frän andra
%Disneyvärldar, inte minst de tecknade längfilmerna. Ytterligare en skillnad är
%att varelserna i vargserierna förefaller stä närmare sina förebilder inom den
%verkliga djurvärlden. Att vargen Zeke vill äta upp grisen Bror Duktig är till
%exempel ett ständigt äterkommande tema, men om katten Svarte Petter skulle fä
%för sig att äta upp musen Musse Pigg skulle detta antagligen höja ett och annat
%ögonbryn bland läsarna.}


% Acknowledgements
% ------------------------------------------------------------------
% Select the language you use in your acknowledgements
\selectlanguage{english}

% Uncomment this line if you wish acknoledgements to appear in the
% table of contents
%\addcontentsline{toc}{chapter}{Acknowledgements}

% The star means that the chapter isn't numbered and does not
% show up in the TOC
\chapter*{Acknowledgements}


-

-
\vskip 10mm

\noindent Espoo, \DATE
\vskip 5mm
\noindent\AUTHOR

% Acronyms
% ------------------------------------------------------------------
% Use \cleardoublepage so that IF two-sided printing is used
% (which is not often for masters theses), then the pages will still
% start correctly on the right-hand side.
\cleardoublepage
% Example acronyms are placed in a separate file, acronyms.tex
% \input{acronyms}

\addcontentsline{toc}{chapter}{Abbreviations and Acronyms}
\chapter*{Abbreviations}

% The longtable environment should break the table properly to multiple pages,
% if needed

\noindent
\begin{longtable}{@{}p{0.25\textwidth}p{0.7\textwidth}@{}}
CD & Contextual Design \\
CI & Contextual Inquiry \\
CMMI & Capability Maturity Model Integration \\
CW & Cognitive Walkthrough\\
ERP & Enterprise Resource Planning \\
HCD & Human-centered design \\
HCI & Human-computer interaction \\
ISI & Interaction Sequence Illustration \\
IT & Information technology \\
SUS & System Usability Scale \\
UI & User interface \\
UMM & Usability Maturity Model \\
UX & User experience \\
SPI & Software process improvement\\


%FLUTE  & The File Delivery over Unidirectional Transport protocol \\
%e.g.& for example (do not list here this kind of common acronymbs or abbreviations, but only those that are essential for understanding the content of your thesis. \\
%note & Note also, that this list is not compulsory, and should be omitted if you have only few abbreviations

\end{longtable}


% Table of contents
% ------------------------------------------------------------------
\cleardoublepage
% This command adds a PDF bookmark that links to the contents.
% You can use \addcontentsline{} as well, but that also adds contents
% entry to the table of contents, which is kind of redundant.
% The text ``Contents'' is shown in the PDF bookmark.
\pdfbookmark[0]{Contents}{bookmark.0.contents}
\tableofcontents

% List of tables
% ------------------------------------------------------------------
% You only need a list of tables for your thesis if you have very
% many tables. If you do, uncomment the following two lines.
% \cleardoublepage
% \listoftables

% Table of figures
% ------------------------------------------------------------------
% You only need a list of figures for your thesis if you have very
% many figures. If you do, uncomment the following two lines.
% \cleardoublepage
% \listoffigures

% The following label is used for counting the prelude pages
\label{pages-prelude}
\cleardoublepage

%%%%%%%%%%%%%%%%% The main content starts here %%%%%%%%%%%%%%%%%%%%%
% ------------------------------------------------------------------
% This command is defined in aalto-thesis.sty. It controls the page
% numbering based on whether the doublenumbering option is specified
\startfirstchapter

% Add headings to pages (the chapter title is shown)
\pagestyle{headings}

% The contents of the thesis are separated to their own files.
% Edit the content in these files, rename them as necessary.
% ------------------------------------------------------------------

% \input{1introduction.tex}

\chapter{Introduction}
\label{chapter:introduction}
This chapter describes the background and reasoning for the thesis as well as the focus and the limitations of the study. This section also presents the research problems and the overall structure of this thesis. 

\section{Motivation and aim of the thesis}
\label{sec:motivationandaim}
In the 1980s, when the usage of personal computers (PCs) became more common, software design practices were still falsely assuming that the users were knowledgeable and competent in computer science. As an outcome, a significant part of the users were practically incapable of using operating systems and applications.During these times, the concepts of Human Computer Interaction (HCI) and usability became important. Since then, the design processes of interactive software for common people has emphasized usability. This process or approach is called human-centered design (HCD). \cite{RefWorks:9}

The term Enterprise Resource Planning (ERP) was invented in the early 1990s.\cite{RefWorks:3} The purpose of the ERP software is to offer techniques and concepts for integrated and thorough management of business, as well as making it more efficient.
The usage of ERP software has increased globally and nowadays even service organizations have invested a lot of resources in ERP implementation.\cite{RefWorks:1, RefWorks:7} 

Despite the importance of the efficiency aspect, the usability of ERP systems is not a widely researched subject area. However, weaknesses in usability may lead into low productivity and make it harder for users to achieve their goals.\cite{RefWorks:2} 

The aim of this thesis is to examine how the usability of a service-oriented ERP system can be enhanced by integrating usability inquiries, inspections and measures into the software development processes. 


This study examines a customer service related business process employed in the subscriber company, (described in more detail in section \ref{sec:background}) and evaluates the state of its usability by using a variety of applicable methods:
\begin{itemize}
\item Contextual Inquiry to define the business process.
\item Cognitive Walkthrough for usability inspection.
\item Interaction Sequence Illustration (ISI) to measure the amount of interaction steps and to understand them.
\item System Usability Scale (SUS) to give a global view of subjective assessments of usability.
\item Remote usability evaluation (by using log data) to evaluate usability in distributed locations.
\end{itemize}

\indent According to the ISO standard of human-centered design for interactive systems \cite{RefWorks:16}, many benefits can be gained by using human-centered methods as a part of software development process. The productivity of an individual user can be increased together with the operational efficiency of an organization. Usable and useful systems also reduce training and help-desk costs, as well as stress and discomfort because they are easily understood by users. In other words, human-centered design improves the UX (User experience). \cite{RefWorks:16}

The benefit of human-centered design (for software development process) is the increased total life cycle of a product and the likelihood of the project succeeding on time and within the budget. The human-centered approach also decreases the risk of software being rejected by the users, or failing to meet the requirements. \cite{RefWorks:16}

\section{Background and research questions}
\label{sec:background}
The subscriber of this thesis is a middle-sized company which is offering information services globally. The company is using in-house software development in order to create dynamic and bespoke software solutions and thereby aiming for commercial efficiency. It is not using any specific software development methodology, but operates in iterative manner. Because of the fast pace of growth, the company is willing to reform their current ERP system as well as the whole software development process. This thesis aims to give answers to the following research questions:

\begin{itemize}
\item \textbf{\emph{How usability methods can help to identify critical disparities in the usage of a system?}}

Understanding the differences in the system usage between individuals can help to understand and deploy best practices throughout the organization and therefore improve efficiency.

\item \textbf{\emph{How much the efficiency of use can be improved by utilizing the results of usability evaluations?}}

It is important to find the most effective and usable user interface solutions and thus decrease the average time spent on tasks. 

\item \textbf{\emph{What usability methods can be practically joined with the software development process of an ERP system?}}

Finding practical and efficient usability methods to be joined with the software development process can improve the quality of end product and also raise the maturity level of the development process.

\end{itemize}



%Usability perspective of the software development process is being considered because of many reasons. There has %been noticeable differences between country offices using the same system and working with the same processes. %Because of the growth of the company, the system must be more efficient, but also pleasant to use.
%\begin{itemize}
%\item Efficiency differences betw

%\end{itemize}

%\begin{itemize}
%                    \item information services
%                    \item publication delivery
%                    \item database services
%                \end{itemize}
%            \item general info about the company
%                \begin{itemize}
%                    \item middle-sized, ~200 employees
%                    \item fast pace of growth
%                    \item personnel service
%                \end{itemize}
%            \item what will be done
%                \begin{itemize}
%                    \item SOA 
%                    \item new process model for software development
%                \end{itemize}
%        \end{itemize}
%    
%    \section{Research problems}
%    \label{sec:researchproblems}
%        \begin{itemize}
%            \item Efficiency differences in country offices.
%            \item Need for more efficient ERP.
%            \item Usability issues
%            \item Can usability methods benefit the understanding of business process
%        \end{itemize}
%    
\section{Scope and structure of the thesis}
\label{sec:scope}

This thesis covers a research study about the usability of a process, which is executed in the ERP system, and employed by the customer service department of the subscriber company. This thesis aims to join the models of software process improvement (SPI) and the human-human centered design to attach usability perspective into the software development of the subscriber company. It covers only the human-centered approach to the development process. The literature  research consists of the SPI, two of its models and a few usability methods. Though the target of the research is ERP software, literature about Enterprise Resource Planning is not covered in the thesis.

The first actual chapter of the thesis elucidates the models of software process improvement. The second chapter describes the usability methods used in this thesis in order to improve the software process. Every usability method used in the research is discussed carefully. The third chapter introduces the process experiment. It covers the experiment steps and reflects the to the process of human-centered design. It also includes the implementation details of the study. The fourth chapter analyzes the data acquired from the process and the implementation process itself. The Last chapter discussed and concludes the research study and the whole thesis.

    
    
%\chapter{Background}
%\label{chapter:background}

%Transitions mentioned in Section~\ref{section:structure} are used also




%\subsection{Finding sources}


%\begin{itemize}
% You can use this command to set the items in the list closer to each other
% (ITEM SEParation, the vertical space between the list items)
%\setlength{\itemsep}{0pt}
%\item Nelli Portal (Aalto Library): \url{http://www.nelliportaali.fi}
%\item ACM Digital library: \url{http://portal.acm.org/}
%\item IEEExplore: \url{http://ieeexplore.ieee.org}
%\item ScienceDirect: \url{http://www.sciencedirect.com/}
%\item \ldots although Google Scholar (\url{http://scholar.google.com/}) will
%find links to most of the articles from the abovementioned sources, if you
%search from within the university network
%\end{itemize}


%information~\cite{howfindinfo}.
%(\url{http://scholar.google.fi/}). It finds academic publications whereas
%\subsection{Referring to sources}
%instructions for many styles~\cite{bibinstructions}. You should ask
%give short examples that are marked with \emph{emphasised text}.
%\emph{Haapasalo~\cite{HaapasaloThesis} researched database algorithms


%If your paragraph has several sources, the above mentioned styles are
%not proper. The reader of your thesis cannot know which of your
%sources give which of the statements. In this case, it is better to
%use more finegraded refering where the reference markings that are
%embedded in the sentences. For example, \emph{the multiversion B+-tree
%  (MVBT) index of Becker et al.~\cite{becker:1996:mvbt} allows database
%  users to query old versions of the database, but the index is not
%  transactional.
%  It's successor, the transactional MBVT (TMVBT), allows a single transaction
%  running in its own thread or process to update the database concurrently
%  with other transactions that only read the
%  database~\cite{haapasalo:2009:tmvbt}.
%  Further development, titled the concurrent MBVT (CMVBT),
%  allows several transactions to perform updates to the database at the same
%  time~\cite{HaapasaloThesis}}.
%  Here, the references are marked before
%  the period in the sentences where they are used.

%  X~[\ldots] according to ms Y~[\ldots] defined that}, if you find a
%\textbf{not} to do it: \emph{\cite{HaapasaloThesis} describes
% \input{3environment.tex}

\chapter{Software process improvement maturity models}
\label{chapter:maturitymodels}

According to O'Regan \cite{RefWorks:29} the software process improvement is "a program of activities designed to improve the performance and maturity of the organization's software processes and the results of such a program". In practice, the aim of SPI is to meet the business goals more efficiently and for example to improve the software quality. In other words, it aims for smarter work and better software in less time. Many process models or frameworks exist for software process improvement and one of them, the Capability Maturity Model Integration (CMMI), is presented in this chapter. Because of the usability approach of this thesis, also a model emphasizing human-centeredness is introduces. The user-centered process model for SPI is called Usability Maturity Model (UMM). \cite{RefWorks:29}

\section{Capability Maturity Model Integration}
The Capability Maturity Model Intergration was developed in the early 1990s by the Software Engineering Institute. Its purpose is to define best practices for software processes in an organization and thereby improve their maturity. In the case of this thesis, the object of interest is the development version of the model, called CMMI for Development (CMMI-DEV). It provides a carefully defined road map and  structured approach for the software process improvement and allows to set improvement goals and priorities. The CMMI consist of five maturity levels and each level includes a number of process areas. These process areas consist of set of goals, which need to be implemented by the defined practices. These practices specify what needs to be done. A maturity level is achieved when all the process areas of that maturity level have been implemented. \cite{RefWorks:29}

After the CMMI is initialized (at first level), the focus at level two is on project management practices such as requirements management and project planning. Level three requires procedures and standards for engineering. For example design, coding and testing should be defined for effective risk management and decision analysis. Process performance must be achieved within the defined limits on the fourth level of the CMMI. The implementation of the level also requires using metrics and setting goals for the performance. The last level of the model requires a culture of continuous improvement in the company. The possible defects need to be identified and actions taken to prevent them to re-occur. Each of the levels and their improvements forms the basis of the next level in the Capability Majority Model. \cite{RefWorks:29}

The level representation of the CMMI is described in Figure \ref{fig:cmmi} including the levels and the CMMI process areas. Every process area consist of \textbf{specific} and \textbf{generic} goals and practices. The specific goals and practices are unique for each process area, and describes what needs to be done to perform the process. The specific practices connected to the specific goals describes the activities to achieve those goals. The generic goals and practices, on the other hand, are common for all the process areas in the CMMI level. The implementation of the generic practices institutionalizes the process, meaning that the process is documented, defined and understood, and that the process users are appropriately trained. The generic goals could be for example to have managed, defined and optimized processes in the organization. \cite{RefWorks:29}	

\begin{figure}[H]
  	\centering
  	\includegraphics[width=0.9\textwidth]{./images/cmmi_levels.png}
  	\caption{Capability Maturity Model Integration: Maturity levels. \cite{RefWorks:29}}
	\label{fig:cmmi}
\end{figure}



\section{Usability Maturity Model}
The Capability Maturity Model Integration is a good example of a model for software process improvement but it doesn't consider the human-centered part of system development. The Usability Maturity Model has been created as a scale to measure the human-centeredness of system development projects. In other words, UMM is a method to evaluate organization's capability to implement human-centered design. It has many corresponding elements with other SPI models, such as ISO TR 15504 standard, but it is still considered a stand-alone model. \cite{RefWorks:30}

 \subsection{Maturity levels of the model}
Like CMMI, also Usability Maturity Model consist of maturity levels (see Figure \ref{fig:umm}). In UMM there are six maturity levels and the first one of them is called level X, in which the need for human-centered activities are not recognized at all. In the level A the organization has recognized the need for improving systems quality of use. Level A1 is a problem recognition attribute and it describes the extent if understanding the problems. To achieve this level of maturity, the staff and the management need to be aware about the need of improvements. Level A2 describes the number of processes performed which provide input for human-centered activities. Information of user requirements should be included in the information collection and practices should be performed to gather this information. In level B, the organization indicates to the staff that the quality of use is considered as important attribute in the development and trains the staff to be aware that usability can be improved by considering the user requirements. The level B1 can be achieved by training the staff human-centered methods and principles of human-system interaction. The staff need focus on the users, for example by understanding the end users' skills, backgrounds and motivations. \cite{RefWorks:30}

The human-centered processes are already implemented in the level C. Users are involved in specifying and testing the system and suitable human-centered techniques are being employed. Active involvement of users, the creation of UX, users defining the quality of use and continuous testing and feedback are required in order for the organization to achieve the level C1 maturity level. The level C2 on the other hand requires the use of appropriate usability methods, suitable facilities and tools, and maintaining usability techniques. Maintaining usability techniques includes reviewing the suitability of the methods and that the state-of-the  art UI technologies are being used. The level C3 can be achieved by ensuring that the staff has the defined and required human-centered competences. In level D and in all the sublevels, the human-centered processes are integrated to the life cycle and quality processes of the system. Also, the required time and resources are targeted for these activities and interaction with other departments is successful, feedback process is administered and the design solutions are being iterated. \cite{RefWorks:30}

 The last level of UMM is level E and its sublevels, which requires the institutionalization of the human-centered processes, meaning that the organization gains benefit from its human-centeredness. On this maturity level, usability skills and engineering skills are used together and the usability defects are managed in a similar manner to other system defects. Also, human-centered process are being included in the projects and usability is systematically improved. Human-centredness has an influence to the whole organization. \cite{RefWorks:30}

\begin{figure}[H]
  	\centering
  	\includegraphics[width=0.7\textwidth]{./images/umm_levels.png}\hspace*{4cm}
  	\caption{Usability Maturity Model: Maturity levels and process attributes. \cite{RefWorks:30}}
	\label{fig:umm}
\end{figure}

\subsection{Level transitions}

The transitions between the maturity levels change the organization and these changes creates the basis for software process improvement. In the case of UMM, transitions represents also the improvement in usability consciousness. 

The transition between maturity levels A and B is a cultural change, from experience based, to more user-centric engineering. In level A, the attitude against user-centeredness might be incredulous, but in level B the awareness about system being used by the people has been created. The transition from level B to C creates the cultural change of user being thought during the development. Also, the differences between analyst and end-users are being recognized. \cite{RefWorks:30}

The Change from the level of considering the users to the implementation of the human-centered processes  (from level C to D) requires the routine use of human factors expertise and human-centered methods and tools. The user involving the development process is considered normal in the level D. The transition from the level D to level E institutionalizes the human-centeredness. The system development is then embedded in a business driven culture, which changes the focus (of the development) from the functionality of the supporting  systems to what the organization is able to do in general. In other words, the system functionality is not the core issue any longer.  \cite{RefWorks:30}

\subsection{Utilization of the model}

A member of staff need to be designated in charge of the quality of use. The first task of this employee is to examine the awareness of the human-centered principles within the organization by interviewing the managerial level. This information can then be used to assess the maturity level of the organization. However, more than one project should be assessed in order to gain wider understanding about the maturity of the organization. The performance assessment forms the basis of plans to review and improve the human-centered processes in the organization. \cite{RefWorks:30}

\chapter{Methods for user-centered software process improvement}
\chaptermark{Methods}
\label{chapter:methods}


In order to be able to discover reliable research data, the research methods must be understood thoroughly. This study gathers data with a few types of usability methods which are selected according to their practicality and utility to the study context. In general, the methods used in this research study can be used as a part of human-centered design process. The reasoning for the choice of the methods is defined in the section \ref{sec:dprocess}. 



\section{Remote usability logging}
\label{sec:rue}

There are many different ways to implement a remote usability evaluation. Typically it is accomplished by surveys or asking for feedback after the system has been deployed. This kind of data is important indicator of user satisfaction, but doesn't really give any specific details about the real system usage. However, the usage data is essential to isolate problems in usability. One approach to access the data remotely is to use data logging method, which is being discussed in this paper. In the context of usability,data logging means practices for mechanically recording the usage of a system. \cite{RefWorks:31}

Usage data of an application can offer valuable information about users' actions and can therefore be utilized in the process of improving software's usability. Even if logging can not replace the traditional usability methods, it provides many advantages over them.  Logging is automatic, objective and it doesn't require direct observation. The data is gathered from the actual running application. \cite{RefWorks:24}	


\subsection{Evaluation process} 

According to Bateman et al. \cite{RefWorks:24} log-based usability evaluation process consist typically of three stages (see Figure \ref{fig:logging_usability_process}). The first stage is called \emph{application instrumentation}. In application instrumentation the logging capability is added to the application. In other words, instrumentation is a process which determines what data will be logged from the usage of an application. In order to gain useful data, successful decisions in instrumentation stage are crucial. Bateman et al. assert that if wrong decisions are made and therefore large amount of low-level data has been collected, the data might be challenging to interpret and might not bring any value. On the other hand, they remark that if only high-level events are logged, internal structures may remain undiscovered. Consequently, both, low-level and high-level usage data need to be tracked and logged. Sometimes, when log data doesn't supply enough information for interpretation, contextual information is needed. Generally, it requires a significant amount of effort and vigilance to be able to gather all the essential data to be analyzed. \cite{RefWorks:24}

\begin{figure}[H]
  	\centering
  	\includegraphics[width=0.8\textwidth]{./images/logging_usability_process.png}
  	\caption{Process for log-based usability evaluation. \cite{RefWorks:24}}
	\label{fig:logging_usability_process}
\end{figure}

The second stage of log-based usability evaluation process is \emph{application usage}. It is a stage where the instrumented application is used in authentic or simulated situation. The log data is collected unobtrusively while the user is performing his or her tasks with the application. \cite{RefWorks:24}

The final stage of the process is \emph{data analysis}, in which a few different approaches can be used \cite{RefWorks:24, RefWorks:25}:
\begin{itemize}
  \item Synchronize and searching
  \item Transforming event streams
  \item Performing counts and summary statistics
  \item Detecting, comparing and characterizing sequences 
  \item Visualizing
\end{itemize}
This part of the process will be described in more detail in the following subsection \nameref{sec:analysis}.



\subsection{Log data analysis}
\label{sec:analysis}
 In order to gain beneficial usability information from user interface events, the log data has to be analyzed thoroughly. According to  Hilbert and Redmiles \cite{RefWorks:25}, there are several approaches to sort out important usability information out of log data and to make it more understandable for humans.
The first one of them is called \textbf{\emph{synchronization and searching}}. 

It is often challenging to interpret user interface events alone and thereby discover valuable higher level events without any contextual information. The purpose of the synchronization and searching is to combine user interface event data with other informative data, such as observations or video recordings, in order to increase the understanding about the context of use. These two forms of data are complementary and can together provide wider understanding about the usability of a system. Observations or video recordings may supply additional information about the user interface events appeared in the log, or vice versa. However, there are few disadvantages with synchronization and searching techniques. For example video recording and observation typically require the presence of the observer. Furthermore, using video recording as a part of evaluation produces a lot of data and can be inefficient to analyze. It might also have an disturbing affect on user's behavior. \cite{RefWorks:25}

The second approach is \textbf{\emph{transformation}}. Transformations combine selection, abstraction and recording to transform events into more beneficial form of information. This information can then be utilized for instance in detecting, comparing or characterizing human behavior patterns. Selection is basically segregating useful information out of mass of user interface related events by filtering out irrelevant data or by selecting the relevant data. Example of selection could be a situation where the user has been writing a lot of text (and thereby generated a lot of data in the log) and is trying to find save button from different menus. From the usability point of view, text inputs as log events are probably not that relevant, but time consuming browsing between different menus can be a sign from usability issues. \cite{RefWorks:25}

Abstraction can be used to combine different events into a more understandable event sequences or patterns. For example, a user typing into different text fields and then clicking a button in a web page could be interpreted as a login activity. However, in this kind of situation user interface event logs must be supplemented with contextual data to be assured that the event sequence is really what it appears to be. Recording means generating new sets of events based on selection and abstraction. Less effort is required to analyze new sets of raw data because earlier analyzing techniques can be exploited. \cite{RefWorks:25}

The third way of generating tangible usability information out of user interface log data is to use \textbf{\emph{counts and summary statistics}}. Counts and summary statistics are calculations based on usability-related metrics gathered from the log data. For instance, calculating the time spent on a specific task (performance time), can be critical usability information. \cite{RefWorks:24, RefWorks:25}

\textbf{\emph{Detecting, comparing and characterizing sequences}} are approaches which utilize sequence information of events. Sequence detection refers to an action in which ready-defined sequences are tried to be identified from the mass of source sequences. Sequence comparison is  executed by usability analyst and it is made between two sequences. These two sequences can be generated for example according to subject or subject groups. In any case, the purpose of sequence comparison is to compare actual usage against some predefined ideal usage. Sequence characterization uses the source sequences to create a model which summarizes all the features of interest in those source sequences. \cite{RefWorks:24, RefWorks:25}
 
The last approach for log-based data analysis is called \textbf{\emph{visualization}}. In visualization, transformed and analyzed data is  presented in a graphical form. General way to visualize data is to use charts, but it is possible to use other visualization techniques such as heat maps based on clicks or mouse traveling. \cite{RefWorks:24, RefWorks:25}

\section{Contextual Inquiry}
\label{sec:cinquiry}
Contextual Inquiry is a qualitative data-gathering and data-analysis methodology. It is adapted from psychology, anthropology and sociology. \cite{RefWorks:28} In practice, Contextual Inquiry is an unstructured interview method, but it has some qualities which differs it from traditional interviews.\cite{RefWorks:23}  It was originally developed to meet three requirements. Firstly it was supposed to identify a design process for systems that will be used similarly in different business contexts and in different cultures. It was also supposed to identify a convenient process for gathering user information in limited time, and finally to identify a way to acquire information about users' work in eligible format. In addition to those requirements, the technique was noticed to be capable of much more. CI cherish participatory design and because of that quality, the users are able to involve in the design. Users contribute to the design by providing a deep understanding about the nature of the work. This is done through inquiry and it is used as a basis for fundamental work concepts. \cite{RefWorks:14} According to Raven and Flanders \cite{RefWorks:28}  The Contextual Inquiry was developed in 1986 at Digital Equipment Corporation by human-computer interaction professionals. \cite{RefWorks:28} 

\subsection{Fundamentals of Contextual Inquiry}
Contextual Inquiry can be said to be an apprenticeship compressed in time. The basis of the method is premised on the idea of user being the expert instead of the interviewer, but unlike an apprentice, the interviewer neither learns the work by doing it, nor has the same amount of time available for learning. \cite{RefWorks:21} CI differs from the traditional master-apprenticeship model in other ways also. A few fundamental principles of Contextual Inquiry are said to be essential in order to meet the specific needs of design problems \cite{RefWorks:21, RefWorks:22}.  These principles are understanding the context of the work, creating a partnership, interpreting the work and steering the focus during the interview. \cite{RefWorks:21, RefWorks:28}

Understanding the \textbf{context} of the work is the baseline of Contextual Inquiry. To gain the understanding about the work structure, the interviewer must pursue understanding about the details of users' work and these details can be found by following the users' actions at work. In general, it is important that interviewer avoids gathering abstract or summarized information about the context.\cite{RefWorks:21}

It is essential to create collaborative environment and a \textbf{partnership} between the user and the interviewer while the real life work structure and activities are tried to be understood thoroughly. Partnership is an equal relationship between the interviewer and the user. In comparison to traditional interview or master-apprentice approach, the partnership doesn't give any power advantage to either of the parties. Instead, it fosters the interviewer's expertise to see the work structure and the user's expertise to do the work. There are many advantages in partnership approach. For example by paying attention to the details and the structure of the work, interviewer can also teach the user to attend to them. In the best case scenario the interviewer and the user contemplates about the work structure and design possibilities together. In this kind of scenario it is common that the work is suspended, while the parties discuss about the work structure, and then return again. For interviewer asking feedback for design ideas is also encouraged.\cite{RefWorks:21}

Even if a partnership should be created between the parties, the interviewer should still be able to steer the interview and keep the \textbf{focus} of the conversation on work-related topics relevant to the design \cite{RefWorks:21}. The focus point should be decided before the research takes place, and data gathering should have a deliberated and precise goal. This goal or focus should depend on the information needs of the design.\cite{RefWorks:22}

The success of Contextual Inquiry and system design depends on the facts gathered, but the facts are not enough. They are a starting point.\textit{"From the fact, the observable event, the designers makes a hypothesis, an initial interpretation about what the fact means or the intent behind the fact."}\cite{RefWorks:21} In other words, \textbf{interpretations} are needed and they are critical for the success of an inquiry. In the final version of the system interpretations have to be correct or the system fails. This is why it is important to share and validate the interpretations with the customer early enough. \cite{RefWorks:21}

There are three types of requirements which should be considered in system development: Technical, business strategic and behavioral. The Contextual Inquiry is a part of Contextual Design (CD), which is a comprehensive design methodology. Contextual Design emerges  within HCI and is used in requirement elicitation focusing on \emph{behavioral requirements}. Therefore, CI can be also used as a practice for examining requirements of a system in authentic environment. \cite{RefWorks:33, RefWorks:36}

As soon as thorough understanding about the work is available, design for a system model can be created.\cite{RefWorks:14} 

\subsection{Contextual Inquiry interviews}

Practical preparations of CI includes careful planning. The first phase of planning is to set the focus for the research. Focus can be for example a definition of a problem which need to be solved. It creates ground rules for the interview and it is therefore easier for the interviewer to steer the conversation. After the focus has been set, the inquiry itself need to be designed. The challenge in inquiry design is to find a way to determine underlying issues which cause problems in the work. The approach for the design should be slightly different if the aim of the Contextual Inquiry is to support upgrading the system, creating a totally new system or redesigning the process. If CI is used to help the design process of a new system, a challenge is to get the designers and the users to work together in order to define new ways of working, and to develop a system design to support them. \cite{RefWorks:21} 

The structure of the interview is considerably straightforward. First task of the interviewer is to introduce the CI process and ask permission to record the conversation and the work. The interviewer has to also make it clear that understanding the work of the user is the primary target of the research, and that all the misunderstandings should be corrected. This is called the conventional interview phase. The next step of the interview is to clarify the rules. In traditional Contextual Inquiry process it is desirable for the interviewer to interrupt and ask questions and correspondingly for the user to indicate if the time is inconvenient for interruption. This phase is called a transition. The third part of the CI is the actual interview (the contextual interview proper phase), which consist of observation, asking direct questions, suggesting interpretations, writing notes and recording the whole chain of events. Finally the interviewer should wrap up (the wrap-up phase) the interview, ensure that everything is understood correctly and summarize the process. This is the last chance for the user to revise misunderstandings. \cite{RefWorks:21}

Contextual Inquiry is usually conducted by one person and the interviewee. If two people are used, the roles of note-taker and interviewer must be separated. This means that interviewer is leading the discussion and the note-taker does not involve in it. The approximate length of the interview should be two hours. It is also important to get an overview of user's background and demographic information in order to be able to focus on relevant things. It is also important \emph{not to use predefined set of questions}, but to familiarize oneself  with the ares of concern in the process. Artifacts offering relevant information about the work, such as cheat sheets or notes, should be also collected, photographed or copied. \cite{RefWorks:27} Reviewing the notes is usually required after the interview in order to ensure their comprehensiveness. It is probable that some ideas or impressions might have forgotten during the interview. \cite{RefWorks:28}


\subsection{Analysing the data}

The data gathered from Contextual Inquiry consist of notes, recordings and possibly artifacts. In order to be able to analyze the data, it has to be in identical format. This is why it might be beneficial to create a summation of the data. It can be done for example by transcribing important notes from the recordings and by describing the artifacts and their use. Once the data is coherent, the analysis can begin. 

The analysis of CI consist of three steps. The first step is to set focus for the analysis. It is often the same than for the inquiry itself, but occasionally insights from the inquiry makes the original focus outdated. In this case, the new focus for the analysis need to be identified. The next phase in CI analysis is to choose the data display. There are various of methods available for that, such as affinity and workflow diagrams. After the data display has been chosen the final step is to organize the data in the data display. \cite{RefWorks:28}

The affinity diagram organizes single notes in to higher categories or hierarchies. There are no predefined categories in affinity diagram for the individual notes. Instead, a single note is being used to define a category and then the corresponding notes are being attached to the same category. In other words, the notes creates categories and categories then raises common structures and themes. After the categories or groups are formed, and there is no floating notes left, the groups have to be descriptively named. Then groups of groups are being collected, and thereby hierarchies created. The named groups and hierarchies, and the headings of them, then represents new information in an affinity. \cite{RefWorks:21, RefWorks:28}

The workflow diagram can be used if the work process need to be tracked and understood thoroughly. At first, the notes from each interview need to be reviewed, after which the flow charts can be conducted. The flow charts should reflect the work process of each individual in every interview. After that all the flowcharts need to be displayed and compared. Finally, a composite work flow diagram (containing the stages perceived in most of the inquiries) can be created. \cite{RefWorks:28}

\subsection{Remote Contextual Inquiry}
The possibility to remotely evaluate the usability of a system was already discussed earlier in this thesis (see section \ref{sec:rue}). Contextual Inquiry can be also implemented remotely with a few modifications on a traditional on-site approach. The remote version of CI is called Remote Contextual Inquiry (RCI). It aims to create a bridge between the users and the developers and is particularly effectively when a project requires feedback from users in distributed locations. RCI can be considered also when the preparation time is limited and the cost is an issue. It can be set up and conducted in less time than the traditional Contextual Inquiry. \cite{RefWorks:32}

In practice, RCI captures the screen of the user working with the software and the usability professional is in contact with the user for example via teleconference and shared screen. Otherwise the activity is basically the same as in traditional CI: User works with the tasks and usability professional observes, records the usage and gathers information about the real-life usage of the system. The analyst also probes and discusses with the user. \cite{RefWorks:32}

The results of the RCI are versatile. Information about the goals of the users and the tasks to achieve those goals are gathered, as well as measurements of the time elapsed and the number of clicks. Moreover RCI provides feedback on layout, content and behavior of the system. \cite{RefWorks:32}
 
\section{Cognitive Walkthrough}
\label{sec:cognitivewalkthrough}

Cognitive Walkthrough (CW) was developed in the early nineties and it was originally intended to help reviewing 'walk-up-and-use' interfaces, such as Automatic Teller Machine (ATM). CW is a formal usability inspection method for professionals involved in the development process. The key concept behind CW is to use theory as a guide for design review. It is easy to understand and apply and therefore feasible to use in a regular development process. \cite{RefWorks:19, RefWorks:18} Contextual Inquiry is additional tool for usability engineering and it is easy to evaluate mockups or sketches of designs relatively quickly with it. \cite{RefWorks:34}

\subsection{Fundamentals of Cognitive Walkthrough}

In theory, human-computer interaction process can be described in four steps. Firstly, the user sets the goal for the activity which is to be accomplished with the system. Secondly, the user examines the user interface for available actions. Thirdly, the user selects the action most likely to make progress. Finally, the user carries out the action and assess the feedback from the system. In real-life tasks these steps would probably iterate to achieve subgoals and to complete tasks. Cognitive Walkthrough examines the correct actions to accomplish a task and the four steps to carry out those actions. \cite{RefWorks:34}

Cognitive Walkthrough consist of two phases: preparatory and analysis phases. The preparatory phase contains prerequisites for the walkthrough. First one of them is a brief description of the user and about the knowledge he or she possess. The second prerequisite is a specific description of one or more tasks to be carried out with the system. The last prerequisite is a list of correct actions required to complete the tasks with the UI. The actual analysis will be accomplished in the analysis phase, where every action of every task will be executed and analyzed. In general the Cognitive Walkthrough method can benefit all phases of system's design and development process.\cite{RefWorks:26, RefWorks:34}

\subsection{Preparations}
Cognitive Walkthrough can be accomplished by using detailed design specification of the user interface which has been received after requirement analysis and functionality definition processes. The walkthrough can be also performed  on a paper simulation, minimal prototype or fully functional prototype of the UI. Formally Cognitive Walkthrough evaluates design's ease of learning by exploring it. \cite{RefWorks:26}

The analysis can be done individually or in a group. In a group the process starts with designers presenting the design to a group of peers and it's usually done after a certain milestone in interface design. The designers can then benefit from the feedback and improve the implementation for the next revision. Participants may represent different organizational units and in the evaluation team they have to adopt different roles, such as recorder, facilitator and various kinds of expert roles. \cite{RefWorks:26}

The first step in walkthrough preparations is to describe the users of the system and choose the tasks for analysis. If the background and technical experience of the users are described in the beginning, more details can be possibly revealed in the walkthrough itself. The selection of tasks (for analysis) is critical and should be based on facts, such as requirement analysis, needs analysis or concept testing. The amount of tasks should be moderate and it is important that the set of chosen tasks include some core functionality and some combinations between those core functionalities. Furthermore, to make tasks as concrete as possible, context descriptions must be included. \cite{RefWorks:26}

After the tasks have been chosen, the action sequences for the tasks need to be described. Basically it means that the sequence of actions, which are required to accomplish the task with the UI, are being described. These actions can be as simple as "press the start button" or "write your name in the text field". However, depending on the level of user expertise and user descriptions, actions might also consist of several simple actions which can then be executed as one block. These kind of actions could be for example filling in the register form or going to a specific website. The interface definition should include the prompts preceding all the actions in the task and the interface's reactions to those actions. If the development is already finished, all the information from the interface is available, but if the development process is only in the beginning, paper descriptions are needed. The level of detail in paper descriptions depend once again on the expertise of the users. \cite{RefWorks:26}

\subsection{Analysis}

The Contextual Inquiry analysis phase examines the actions of the tasks and generates a plausible story or a review about the reasons why the users (which have been defined earlier) would have chosen those actions. These stories are based on presumptions about user's expertise and objectives. Sometimes users trust on their problem-solving skills, which is why it is important to understand the problem-solving process in the analysis phase. In order to mimic this process in the analysis phase, four steps should be taken. First, a rough description of the task to be accomplished should be considered. Then the user interface should be explored and actions should be taken according to assumptions users might have. The third step is to observe if the user interface is returning the expected results for each action. The last part of problem-solving process is to assume and define users' next action.\cite{RefWorks:26} In general, the walkthrough or the analysis evaluates if the user is able to select the correct actions with the user interface.

Four criteria can be used in order to assess the ease of performing the correct actions and thereby completing the task. According to the criteria, the goal, the accessibility of the correct user interface object, the match between the label of the object and the object itself, and the feedback provided should be considered while evaluating the system. \cite{RefWorks:34}
 
\section{Interaction Sequence Illustration}
\label{sec:isi}
Interaction Sequence Illustration is a modified usability inspection method. It differs from traditional inspection methods, such as Cognitive Walkthrough and heuristic evaluation, in a significant manner. Unlike other inspection methods, ISI does not function in isolation from system's actual context and users. It conducts the model of interaction from the real-life environment. In case multiple systems are required to perform the user's task, ISI can also focus on many systems instead of just one. \cite{RefWorks:17} In general, ISI combines user-based testing and usability inspection approaches. The method was originally developed for evaluating the usability of Information Technology (IT) tasks carried out in healthcare industry, but there are no defined reasons why it could not be utilized in different environments.

\subsection{Description of the Interaction Sequence Illustration}
Interaction Sequence Illustration focuses on low level analysis of human-computer interaction and exploits the data acquired during the Contextual Inquiry process. However, the inquiry data has to be complemented with documentation about interaction activities, photos and screenshots. There are three objectives for ISI method to handle. The first is to demonstrate how the user perceives the system. The second is to identify and document activities, and to discover problems. The first two objectives forms the basis for the third and more extensive objective, which is to support the user-centered design and development. \cite{RefWorks:17}  The strength of ISI lies in the analysis of data and it can be used to compare the interaction sequences of two of or more UI implementations. On the other hand it can 		provide prominent information on only one UI's interaction sequence. 

ISI generates two analysis from the collected data. The first one is an analysis of interaction stages, which divides the whole interaction sequence in to main phases or stages. This analysis is performed based on the inquiries. The second analysis is so called step-by-step illustration. It defines the stages and interaction steps by numbers, photos, descriptions and screenshots. \cite{RefWorks:17} It is basically a well defined and ordered workflow description. 

\subsection{Utilization of the Interaction Sequence Illustration}
The utilization of Interaction Sequence Illustration can be simplified in seven steps (see Figure \ref{fig:isi_chart}). First the data need to be collect alongside Contextual Inquiry interview, including inquiry data, possible photos, screenshots and notes.
After the data collection the screenshots need to be arranged in the right order and all the superfluous data need to be removed. The third phase is to count the interaction steps based on the screenshots and activity analysis, and to organize interaction steps into stages.
Next the screenshots need to be modified and important details highlighted. Finally, the sequence numbers and  detailed description texts should be attached to the screenshots in order to give a profound understanding about the actions. The outcome of the method is an extensive illustration of the interaction sequence, information about the usability of the system and about it's effectiveness of use. \cite{RefWorks:17}
	
\begin{figure}[H]
  	\centering
  	\includegraphics[width=1.1\textwidth]{./images/isi_chart.png}
  	\caption{Interaction Sequence Illustration stages and outcomes.}
	\label{fig:isi_chart}
\end{figure}

\section{System Usability Scale}
\label{sec:sus}
In his paper Brooke \cite{RefWorks:10} argued that usability is not any real existing quality, but a good usability artifact is \emph{appropriate to its purpose}. In other words "the usability of any tool or system has to be viewed in terms of the context in which it is 			used, and its appropriateness to that context"\cite{RefWorks:10}. Still in many cases, context related usability evaluation is not desirable. The reason for this is that a large scale context analysis is usually neither cost-efficient nor practical.\cite{RefWorks:10} In the 1986, System Usability Scale was introduced to respond these challenges by offering "quick and dirty"\cite{RefWorks:10} way to get subjective ratings about the usability of a system. \cite{RefWorks:12, RefWorks:35} System Usability Scale can be also used as a supplement for another usability evaluation methods.

\subsection{Description of the System Usability Scale}
System Usability Scale (see \nameref{app:susform}) is a ten-item \emph{Likert scale}, meaning that every item consist of the scale of five, ranging from "Strongly disagree" to "Strongly agree". The questionnaire is generally being filled right after the possibility to use 		the system to be evaluated. The focus should be on immediate responses and too much time shouldn't be given to the respondents. \cite{RefWorks:10} One of the best qualities of SUS is that it's not dependent on any specific technology or user interface type. Thereby it can be used to evaluate the usability of wide range of different kinds of interfaces from traditional desktop user interfaces to mobile web UIs. SUS is also relatively fast to implement by administrators and to use by the participants of the study. \cite{RefWorks:12}
\texttt{TÄHÄN LISÄÄ!!!rw12 -ARTIKKELISTA}


\subsection{System Usability Scale in practice}
The outcome of SUS is a single value which express the overall usability of the system. The value of the method consist of all the items and none of them are meaningful as such. System Usability Scale can be calculated by summing the score contribution (range from 0 to 4) 		from each item. Before summing the scale positions, the items 1,3,5,7 and 9 need to be subtracted by one and the items 2,4,6,8 and 10 need to be subtracted from 5. The last step is to multiply the sum of the scores by 2.5 to get the overall SUS value, which will range from minimum of 0 to maximum of 100. \cite{RefWorks:10} The resulting single score is an easy-to-understand measure, and can therefore be discussed with the wide range of stakeholders. \cite{RefWorks:12} 
    
\chapter{Implementation of the study}
\label{chapter:implementation}

According to ISO standard 9241-210, human-centered design consist a few activities and iterative process (see Figure \ref{fig:hci_process}). \cite{RefWorks:16} The empirical part of this thesis adapts human-centered design principles and the steps of the process experiment are highly linked to its activities. The rationale for adopting human-centered design is to experiment how standardized human-centered activities fit in with company's software development process. The human-centered design and the selected methods are represented as a part of software development process in figure \ref{fig:lm_process}. It also defines the fundamental idea behind the thesis: Trying to improve in-house software development process by introducing human-centered activities and methods.

This chapter contains description about the principles of human-centered design. Applied steps utilized in the research study are described in section \ref{sec:dprocess}. This chapter will also describe the implementation phases of the experiment. 

\begin{figure}[H]
  	\centering
    	\includegraphics[width=1.0\textwidth]{./images/lm_process.png}
  	\caption{Human-centered design as a part of in-house software development process.\cite{RefWorks:16}}
	\label{fig:lm_process}
\end{figure}


\section{Principles of human-centered design}
Human-centered design doesn't require any particular design process, and it functions as a supplement to existing design methodologies. However, there are some principles which should be followed in order the process to be human-centered. 

Firstly, the understanding of users, tasks and environments should create the basis for the design. This means that all the user groups and stakeholders who might be affected directly or indirectly by the developed system, should be considered. 

Secondly, users should be actively involved in the design and development processes, in order to provide crucial information about the context of use and the practices of work. The people participating in the design and development processes should possess comprehensive understanding about the work and should be able to represent wide range of users. 

Thirdly, the design should be driven and refined by user-centered evaluations. This means that the feedback from user evaluations should be used as a basis for improving the design. This approach efficiently minimizes the risk of system not meeting the requirements, including hidden and implicit requirements. 

Fourthly, the process should be iterative. Defining all the details and user requirements is impossible in the beginning of the system development. This is why iterative revising and refining of the descriptions, specifications and prototypes is required. Iterative development also eliminates uncertainty during the development.

USER'S STRENGTHS LIMITATIONS .......


\section{Applied human-centered design process}
\label{sec:dprocess}

The subscriber company has a local customer service department in every country it operates and they are in charge of customer creation process to the ERP system. The employees of customer services are also maintaining customer information in order to keep them up to date, which requires daily actions to the system, such as editing delivery addresses and invoicing frequencies. Furthermore, the accuracy of customer information in the system is also highly important for the business. Consequently, the customer information processing is examined throughly and used as an example to improve the existing software development process.

\begin{figure}[H]
  	\centering
    	\includegraphics[width=1.0\textwidth]{./images/hci_process.png}
  	\caption{Human-centered design activities.\cite{RefWorks:16}}
	\label{fig:hci_process}
\end{figure}



This part of the thesis forms the first activity of the human-centered design called \emph{planning the design process}. The section applies the activities of the human-centered design and creates a proposal of methods and procedures to be included in the company's software development process. 

The examined ERP process is executed by the personnel highly expertized in the area of customer service and the primary tool for the process is a tailored ERP system. Moreover, the distinctive nature of the business sets additional challenges for the usability evaluation. This is why it is presumable that the process can not be understood thoroughly without understanding the work in practice. Because of the reasons mentioned earlier, Contextual Inquiry is chosen as a method for \emph{understanding and specifying the context of use}. 
Also, the researcher must have a profound understanding about the context of use, in order to \emph{specify the user requirements}. A qualified system can be implemented only if the user requirements are precisely defined. Many of the methods used in this experiment aims to assist in this crucial state of development. Technically, the process of requirement specification can be initiated while running the Contextual Inquiry. Although, it is essential to remember that the focus in CI should be mostly on understanding the context of use. This thesis uses user evaluation, remote usability evaluation, expert evaluation and quantitative metrics to identify all the user requirements affecting on the user experience and the functionality of the system. 

\texttt{TÄHÄN SIITÄ ETTÄ CI TOTEUTETTIIN RCI:nä}

Interaction Sequence Illustration method is used in the thesis to identify and represent the interaction steps required to accomplish the customer creation process to the system. The method was chosen because one of the main objectives of this thesis \emph{is to find out how much the efficiency of use can be affected by utilizing the results of usability evaluation}. ISI can be used not only to identify all the interaction steps, but to detect the unnecessary steps as well. If the unnecessary phases can be excluded from the process, it might have significant affects on the efficiency of use and reveal tacit user requirements. 

The user experience might have influence on the efficiency of use. This is why the UX of the system should be considered even if the company's distinctive ERP system is developed for the use of experts. The System Usability Scale is being used to understand the level of user satisfaction.

Traditional and continuous usability evaluation would be difficult and expensive for the subscriber company because of its distributed operational environment. Hence, the possibility of accommodating remote usability evaluation to the company's software development process is examined in this thesis. Remote usability evaluation in this context means the remote usability related user interface data logging. The desirable results from using the data logging would be utilizable, location independent usability data, which could help the development of the global ERP system and reveal user requirements. 

Expert evaluation is utilized using Cognitive Walkthrough. The objective of CW in the thesis is to identify user requirements which are not distinctive and cannot be perceived with the user evaluations. Cognitive Walkthrough is used to ensure that any crucial requirements will not stay unnoticed.

After the user requirements have been specified, \emph{design solution can be produced according to these requirements}. Analyzed data is applied to create a new user-centered prototype solution. The solution and \emph{the design is then evaluated against the requirements}. 

Figure \ref{fig:hcd_process} illustrates the methods used and the actions taken in the applied design process and reflects them against the actual human-centered design process. Contextual Inquiry, Interaction Sequence Illustration and System Usability Scale are utilized to understand the context of use and the level of sophistication of the current solution. Those methods in addition to Cognitive Walkthrough are applied to specify the user requirements. The SUS and ISI methods measures the existing solution and the data is exploited in analysis and user requirement specification. The design solutions and the prototype is evaluated with Cognitive Walkthrough, ISI, remote usability data logging and System Usability Scale. The final analysis is carried out using the data from the second evaluation.  

\texttt{COULD BE A PART OF AGILE/ITERATIVE DEVELOPMENT}

\begin{figure}[H]
  	\centering
    	\includegraphics[width=1.0\textwidth]{./images/applied_hcd_process.png}
  	\caption{Relations between human-centered design process stages, methods utilized in the thesis and the implementation stages of the thesis.}
	\label{fig:hcd_process}
\end{figure}

\section{Practical implementation}
 \label{sec:implementation}
    
 The practical implementation of the research study was dated between August and NOVEMBER???? 2013. Altogether six users from two different countries participated in remote data logging, Contextual Inquiry and Interaction Sequence Illustration evaluations. They also filled in the System Usability Scale form. Cognitive Walkthrough was implemented by one usability analyst. The users' work experience varied from under a year to a quarter of a century. In every evaluation session the users were situated in their own workstations. 
	\texttt{Kuinka monta osallistujaa, millainen ympäristö, missä maassa, minkä ikäisiä osallistujat, kauanko tehneet työtä}
    	\texttt{kuinka testaus sujui, menetelmäkohtaisesti, mitä välineitä käytettiin miten toimi.}
    	\\*
    	-kaksi vaihetta: ennen ja jälkeen prototyypin
    	\\*
    	-vaihe1: testi-tietokanta käytössä, jotta pääsi paremmin käsiksi toimintaan, osittain myös tuotanto-kannan kanssa tekemistä, jotta autentisuus saatiin esille (tarkemmat tsekkaukset kirjoituksessa yms...).
    	\\*
    	 -neljä osallistujaa suomesta
    	 \\*
    	 -vuodet yrityksessä (noin): 1,21,14,0.5 = avg.9
    	 \\*
    	 -suomessa työympäristö
\subsection{Contextual Inquiry}
Contextual Inquiry was executed in three separated phases in the research study. In the first CI implementation the user and the analyst shared the common space, in this case the user's workstation. 
-ISI ja ci toteutettu  kaksi kertaa etänä.
 \\*
	  
	 -kesto noin kaksi tuntia
	  \\*
	  -eteneminen: ensin ci ja lopuksi määritelty interaktio-setti isiä varten.
	  \\*
	   ymmärrys järjestelmästä ja prosessista. käyttäneet paljon, päivittäin.
	  \\*
	  -analyysi toteutettiin affiniteettidiagrammin avulla.
	  
	  \subsection{Interaction Sequence Illustration}

	  -vaihe1 ja 2:toteutettiin interaktio-setin kautta
	  \\*
	  -pyrittiin mahdollisimman laaja-alaiseen täyttöön esimerkki-tehtävässä.
	  \\*
	  -analyysi toteutettiin menetelmän ohjeiden mukaisesti.
	
	  \subsection{Cognitive Walkthrough}
	  \subsection{System Usability Scale}
	  -toteutettiin nettilomakkeella anonyymisti
	  \\*
	  -osallistumisprosentti.
	  \\*
	  analyysi vanhaan tutkimukseen peilaten.
	  \subsection{Remote usability evaluation}
	  -results 
	  -no of clicks 
	\subsection{Building the prototype}
	-technology: .NET, C, WPF, WCF
	\\
	-using the results from the first round evaluation.
	\\
	-limitations
	\\
	-functionality
    	
\chapter{Analysis}
\label{chapter:analysis}

    \section{Results and comparisons}
    \label{sec:results}
    
    \subsection{Overview}
    -sales department's contribution is required.
    
    \subsection{Contextual Inquiry}
    vaihe1: suomessa:57 huomiota, ongelmakohtaa,parannusideaa 
    \\*
    -6 kategoriaa
    \\*

    \subsection{System Usability Scale}
    -google form
    \\*
     -vaihe1: suomessa tulos: 48.125/100
     \\*
     -vaihe1: suomessa tulos ei välttämättä kerro koko totuutta koska otanta kohtalaisen pieni.
     \\*
     
     \subsection{Interaction Sequence Illustration}
     	  -interaktioiden ajallinen mittaaminen ei onnistu
	  \\*
	  -interaktioiden määrä vaihtelee riippuen tilanteesta.aiheutti hankaluuksia, joten otettu keskiarvo.s
	  \\*
	  -mitkä ovat interaktio stepit ja staget.
	  \\*
	  -vaihe1: interaktio katsottu videolta ja laskelmat tehty olemassa oleviin asiakkuuksiin syötettyjen tietoihin perustuen. 50 satunnaisotanta antanee kohtalaisen selkeän kuvan siitä kuinka paljon prosessi vaatii interaktioita.
	  \\*
	  -vaihe 1: järjestelmä antaa periaatteessa mahdollisuuden täyttää 83 kenttää per asiakastyyppi + aloitus stepit -> käytännössä näin ei kuitenkaan ole.
	  \\*
	  -vaihe1:sopimusasiakkaalla 50 satunnaisotannan avg. steppien määrä on 23, pienin 9 ja suurin 35 (laskettu tietojen syötöstä  tallenna-klikkaukseen.)
	  \\*
	  -vaihe1:laskutusasiakkalla 50 satunnaisotannan avg. steppien määrä on 14,48, pienin 9 ja suurin 22 (laskettu lisää laskutusasiakas -painikkeen klikkauksesta tallentamiseen)
	  \\*
	  -vaihe1:toimitusasiakkaalla 50 satunnaisotannan avg. steppien määrä on 12,94, pienin 11 ja suurin 18 (laskettu lisää toimitusasiakas -painikkeen klikkauksesta tallentamiseen)
	  \\*
	  -vaihe1:uudistusasiakkaalla 50 satunnaisotannan avg. steppien määrä on 15,66 pienin 11 ja suurin 24 (laskettu lisää uudistusosoite -painikkeen klikkauksesta tallentamiseen). myös laskutusasiakas voi olla uudistusasiakas, mikä voi vääristää tulosta hieman.
	  \\*
	  -vaihe1:yhteensä asiakasinteraktioiden avgt on 66,08 steppiä. tähän tulee kuitenkin lisätä vielä libsysin käynnistykseen liittyvät stepit.
	  \\*
     
     The results of the research may not be suitable for every organization.
    -Comparison between country offices, between methods
    -Comparison between individuals (esim. kuinka kauan kesti tietyn toiminnon tekeminen) / Overall comparison (esim. kaikkien koehenkilöiden yhteinen kehitys.)
- Interaction sequence illustration (esim. kuinka monta steppiä ennen ja jälkeen)

    \section{Implementation analysis}
    \label{sec:implementationanalysis}
    -Should these methods be implemented as a part of the process or not.
    -While writing the thesis, contextual inquiry used as part of requirement definition process.

%When you use \texttt{pdflatex} to render your thesis, you can include PDF images
%directly, as shown by Figure~\ref{fig:indica_model} below.

%\begin{figure}[ht]
%  \begin{center}
%    \includegraphics[width=\textwidth]{example_indica_model.pdf}
%    \caption{The INDICA two-layered value chain model.}
%    \label{fig:indica_model}
%  \end{center}
%\end{figure}

%You can also include JPEG or PNG files, as shown by Figure~\ref{fig:eeyore}.

%\begin{figure}[ht]
%  \begin{center}
%    \includegraphics[width=9cm]{example_ihaa.jpg}
%    \caption{Eeyore, or Ihaa, a very sad donkey.}
%    \label{fig:eeyore}
%  \end{center}
%\end{figure}


%If you have PS or EPS files, you can use the tools \texttt{ps2pdf} or
%\texttt{epspdf} to convert your PS and EPS files to PDF\@.

% Comment: If your sentence ends in a capital letter, like here, you should
% write \@ before the period; otherwise LaTeX will assume that this is not
% really an end of the sentence and will not put a large enough space after the
% period. That is, LaTeX assumes that you are (for example), enumerating using
% capital roman numerals, like I. do something, II. do something else. In this
% case, the periods do not end the sentence.

% Similarly, if you do need a normal space after a period (instead of
% the longer sentence separator), use \  (backslash and space) after the
% period. Like so: a.\ first item, b.\ second item.


%WYSIWYG vector editor that allows you to save directly to PDF\@.

%Excel to PDF format, and then add them; or use \texttt{gnuplot}, which can
%something like \texttt{set term pdf \ldots}.


%rather steep learning curve. Locate the manual (\texttt{pgfmanual.pdf}) from
%graphics is shown in Figure~\ref{fig:page-merge}.

%\begin{figure}[ht]
%  \begin{center}
%    \input{example_page-merge.tex}
%    \caption{Example of a multiversion database page merge. This figure has
%    been taken from the PhD thesis of Haapasalo~\cite{HaapasaloThesis}.}
%    \label{fig:page-merge}
%  \end{center}
%\end{figure}


% These definitions are only used in the example images; you will not
% need them for your thesis...
%\newlength{\graphdotsize}
%\setlength{\graphdotsize}{1.7pt}
%\newlength{\graphgridsize}
%\setlength{\graphgridsize}{1.2em}
%\begin{figure}[ht]
%\begin{center}
%\subfigure[Examples of obstruction graphs for the Ferry Problem]{
%  \input{example_obstruction-grouped.tex}
%}
%\subfigure[Examples of star graphs]{
%  \input{example_general-star-graphs.tex}
%}
%\caption{Examples of graphs draw with TikZ. These figures have been taken from a
%course report for the graph theory course~\cite{FerryProblem}.}
%\label{fig:tikz-examples}
%\end{center}
%\end{figure}



% \input{4methods.tex}


% An example of a traditional LaTeX table
% ------------------------------------------------------------------
% A note on underfull/overfull table cells and tables:
% ------------------------------------------------------------------
% In professional typography, the width of the text in a page is always a lot
% less than the width of the page. If you are accustomed to the (too wide) text
% areas used in Microsoft Word's standard documents, the width of the text in
% this thesis layout may suprise you. However, text in a book needs wide
% margins. Narrow text is easier to read and looks nicer. Longer lines are
% hard to read, because the start of the next line is harder to locate when
% moving from line to the next.
% However, tables that are in the middle of the text often would require a wider
% area. By default, LaTeX will complain if you create too wide tables with
% ``overfull'' error messages, and the table will not be positioned properly
% (not centered). If at all possible, try to make the table narrow enough so
% that it fits to the same space as the text (total width = \textwidth).
% If you do need more space, you can either
% 1) ignore the LaTeX warnings
% 2) use the textpos-package to manually position the table (read the package
%    documentation)
% 3) if you have the table as a PDF document (of correct size, A4), you can use
%    the pdfpages package to include the page. This overrides the margin
%    settings for this page and LaTeX will not complain.
% ------------------------------------------------------------------
% Another note:
% ------------------------------------------------------------------
% If your table fits to \textwidth, but the cells are so narrow that the text
% in p{..}-formatted cells does not flow nicely (you get underfull warnings
% because LaTeX tries to justify the text in the cells) you can manually set
% the text to unjustified by using the \raggedright command for each cell
% that you do not want to be justified (see the example below). \raggedleft
% is also possible, of course...
% ------------------------------------------------------------------
% If you need to have linefeeds (\\) inside a cell, you must create a new
% paragraph-formatting environment inside the cell. Most common ones are
% the minipage-environment and the \parbox command (see LaTeX documentation
% for details; or just google for ``LaTeX minipage'' and ``LaTeX parbox'').
%\begin{table}
%\begin{tabular}{|p{2cm}|p{3.8cm}|p{4.5cm}|p{1.1cm}|}
% Alignment of sells: l=left, c=center, r=right.
% If you want wrapping lines, use p{width} exact cell widths.
% If you want vertical lines between columns, write | above between the letters
% Horizontal lines are generated with the \hline command:
%\hline % The line on top of the table
%\textbf{Code} & \textbf{Name} & \textbf{Methods} & \textbf{Area} \\
%\hline
% Place a & between the columns
% In the end of the line, use two backslashes \\ to break the line,
% then place a \hline to make a horizontal line below the row

%    Software & \raggedright Computer simulations, mathematical modeling,

%\hline
%\multicolumn{2}{|p{6.25cm}|}{Mat-2.3170 Simulation (here is an example of
% multicolumn for tables)}& Details of how to build simulations & T-110 \\
% The multicolumn command takes the following 3 arguments:
% the number of cells to merge, the cell formatting for the new cell, and the
% contents of the cell
%\hline
%S-38.3184 & Network Traffic Measurements and Analysis
%& \raggedright How to measure and analyse network
%  traffic & T-110 \\ \hline
%\end{tabular} % for really simple tables, you can just use tabular
% You can place the caption either below (like here) or above the table
%\caption{Research methodology courses}
% Place the label just after the caption to make the link work
%\label{table:courses}
%\end{table} % table makes a floating object with a title



% \input{5implementation.tex}


% \input{6evaluation.tex}



%You have done your work, but that's\footnote{By the way, do \emph{not} use







% \input{7discussion.tex}

\chapter{Conclusions and discussions}
\label{chapter:conclusion}

jos käytetään hcd:tä ja menetelmiä voidaan päästä eteenpäin umm tasoilla.??
        \section{Future work}
	\label{sec:future}
	-need for more simple methods and more versatile data.
%At this point, you will have some insightful thoughts on your implementation
%and you may have ideas on what could be done in the future.
%This chapter is a good place to discuss your thesis as a whole and to show your
%professor that you have really understood some non-trivial aspects of the
%methods you used\ldots



% \input{8conclusions.tex}


%Time to wrap it up!
%Write down the most important findings from your work.
%Like the introduction, this chapter is not very long.
%Two to four pages might be a good limit.
    


% Load the bibliographic references
% ------------------------------------------------------------------
% You can use several .bib files:
% \bibliography{thesis_sources,ietf_sources}
\bibliography{ref}


% Appendices go here
% ------------------------------------------------------------------
% If you do not have appendices, comment out the following lines

% \input{appendices.tex}
%\chapter{SUS form}
%\label{chapter:susform}

\appendix
\includepdf[scale=0.9,pages=4,angle=0.1,picturecommand*={%
     \put(110,770){%
         \parbox{\textwidth}{\chapter{SUS form}\label{app:susform}}
     }}]{./pdfs/sus.pdf}


\includepdf[scale=0.85,pages=1,angle=0.0,picturecommand*={%
     \put(95,780){%
         \parbox{\textwidth}{\chapter{Contextual Inquiry: Research plan}\label{app:ciresearch}}
     }}]{./pdfs/ciplan.pdf}



%\includepdf[scale=0.9,pages=4,picturecommand*={%
%     \put(110,770){%
%         \parbox{\textwidth}{\chapter{TEST form}\label{app:susform}}
%     }}]{sus.pdf}


%\includegraphics[scale=0.5,pages={4}]{sus.pdf}

%\includepdf[ pages=4, scale=.7, frame, pagecommand =  \chapter{}]{sus.pdf}
%\chapter{}
%\includegraphics[page=4,width=1.2\textwidth]{sus.pdf}
%\chapter{SUS form}
%\label{chapter:susform}
%\begin{figure}[h]
 %  \centering
   %\begin{tabular}{@{}c@{\vspace{2.5cm}}c@{}}
      % \includegraphics[page=4,width=1.2\textwidth]{sus.pdf} & 

   %\end{tabular}
 %\label{fig:Test}
%\end{figure}



%    \section{APPENDIX A}
%    \label{sec:appendixa}
%This is the first appendix. You could put some test images or verbose data in an
%appendix, if there is too much data to fit in the actual text nicely.

%For now, the Aalto logo variants are shown in Figure~\ref{fig:aaltologo}.

%\begin{figure}
%\begin{center}
%\subfigure[In English]{\includegraphics[width=.8\textwidth]{aalto-logo-en}}
%\subfigure[Suomeksi]{\includegraphics[width=.8\textwidth]{aalto-logo-fi}}
%\subfigure[Pä svenska]{\includegraphics[width=.8\textwidth]{aalto-logo-se}}
%\caption{Aalto logo variants}
%\label{fig:aaltologo}
%\end{center}
%\end{figure}


% End of document!
% ------------------------------------------------------------------
% The LastPage package automatically places a label on the last page.
% That works better than placing a label here manually, because the
% label might not go to the actual last page, if LaTeX needs to place
% floats (that is, figures, tables, and such) to the end of the
% document.
\end{document}
