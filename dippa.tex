% Lines starting with a percent sign (%) are comments. LaTeX will
% not process those lines. Similarly, everything after a percent
% sign in a line is considered a comment. To produce a percent sign
% in the output, write \% (backslash followed by the percent sign).
% ==================================================================
% Usage instructions:
% ------------------------------------------------------------------ 
% The file is heavily commented so that you know what the various
% commands do. Feel free to remove any comments you don't need from
% your own copy. When redistributing the example thesis file, please
% retain all the comments for the benefit of other thesis writers!
% ==================================================================
% Compilation instructions:
% ------------------------------------------------------------------
% Use pdflatex to compile! Input images are expected as PDF files.
% Example compilation:
% ------------------------------------------------------------------
% > pdflatex thesis-example.tex
% > bibtex thesis-example
% > pdflatex thesis-example.tex
% > pdflatex thesis-example.tex
% ------------------------------------------------------------------
% You need to run pdflatex multiple times so that all the cross-references
% are fixed. pdflatex will tell you if you need to re-run it (a warning
% will be issued)
% ------------------------------------------------------------------
% Compilation has been tested to work in ukk.cs.hut.fi and kosh.hut.fi
% - if you have problems of missing .sty -files, then the local LaTeX
% environment does not have all the required packages installed.
% For example, when compiling in vipunen.hut.fi, you get an error that
% tikz.sty is missing - in this case you must either compile somewhere
% else, or you cannot use TikZ graphics in your thesis and must therefore
% remove or comment out the tikz package and all the tikz definitions.
% ------------------------------------------------------------------

% General information
% ==================================================================
% Package documentation:
%
% The comments often refer to package documentation. (Almost) all LaTeX
% packages have documentation accompanying them, so you can read the
% package documentation for further information. When a package 'xxx' is
% installed to your local LaTeX environment (the document compiles
% when you have \usepackage{xxx} and LaTeX does not complain), you can
% find the documentation somewhere in the local LaTeX texmf directory
% hierarchy. In ukk.cs.hut.fi, this is /usr/texlive/2008/texmf-dist,
% and the documentation for the titlesec package (for example) can be
% found at /usr/texlive/2008/texmf-dist/doc/latex/titlesec/titlesec.pdf.
% Most often the documentation is located as a PDF file in
% /usr/texlive/2008/texmf-dist/doc/latex/xxx, where xxx is the package name;
% however, documentation for TikZ is in
% /usr/texlive/2008/texmf-dist/doc/latex/generic/pgf/pgfmanual.pdf
% (this is because TikZ is a front-end for PGF, which is meant to be a
% generic portable graphics format for LaTeX).
% You can try to look for the package manual using the ``find'' shell
% command in Linux machines; the find databases are up-to-date at least
% in ukk.cs.hut.fi. Just type ``find xxx'', where xxx is the package
% name, and you should find a documentation file.
% Note that in some packages, the documentation is in the DVI file
% format. In this case, you can copy the DVI file to your home directory,
% and convert it to PDF with the dvipdfm command (or you can read the
% DVI file directly with a DVI viewer).
%
% If you can't find the documentation for a package, just try Googling
% for ``latex packagename''; most often you can get a direct link to the
% package manual in PDF format.
% ------------------------------------------------------------------


% Document class for the thesis is report
% ------------------------------------------------------------------
% You can change this but do so at your own risk - it may break other things.
% Note that the option pdftext is used for pdflatex; there is no
% pdflatex option.
% ------------------------------------------------------------------
\documentclass[12pt,a4paper,oneside,pdftex]{report}

% The input files (tex files) are encoded with the latin-1 encoding
% (ISO-8859-1 works). Change the latin1-option if you use UTF8
% (at some point LaTeX did not work with UTF8, but I'm not sure
% what the current situation is)
\usepackage[utf8]{inputenc}
% OT1 font encoding seems to work better than T1. Check the rendered
% PDF file to see if the fonts are encoded properly as vectors (instead
% of rendered bitmaps). You can do this by zooming very close to any letter
% - if the letter is shown pixelated, you should change this setting
% (try commenting out the entire line, for example).
\usepackage[OT1]{fontenc}
% The babel package provides hyphenating instructions for LaTeX. Give
% the languages you wish to use in your thesis as options to the babel
% package (as shown below). You can remove any language you are not
% going to use.
% Examples of valid language codes: english (or USenglish), british,
% finnish, swedish; and so on.
\usepackage[finnish,swedish,english]{babel}

\usepackage{enumerate}

% Font selection
% ------------------------------------------------------------------
% The default LaTeX font is a very good font for rendering your
% thesis. It is a very professional font, which will always be
% accepted.
% If you, however, wish to spicen up your thesis, you can try out
% these font variants by uncommenting one of the following lines
% (or by finding another font package). The fonts shown here are
% all fonts that you could use in your thesis (not too silly).
% Changing the font causes the layouts to shift a bit; you many
% need to manually adjust some layouts. Check the warning messages
% LaTeX gives you.
% ------------------------------------------------------------------
% To find another font, check out the font catalogue from
% http://www.tug.dk/FontCatalogue/mathfonts.html
% This link points to the list of fonts that support maths, but
% that's a fairly important point for master's theses.
% ------------------------------------------------------------------
% <rant>
% Remember, there is no excuse to use Comic Sans, ever, in any
% situation! (Well, maybe in speech bubbles in comics, but there
% are better options for those too)
% </rant>

% \usepackage{palatino}
% \usepackage{tgpagella}



% Optional packages
% ------------------------------------------------------------------
% Select those packages that you need for your thesis. You may delete
% or comment the rest.

% Natbib allows you to select the format of the bibliography references.
% The first example uses numbered citations:
%\usepackage[square,sort&compress,numbers]{natbib}
% The second example uses author-year citations.
% If you use author-year citations, change the bibliography style (below);
% acm style does not work with author-year citations.
% Also, you should use \citet (cite in text) when you wish to refer
% to the author directly (\citet{blaablaa} said blaa blaa), and
% \citep when you wish to refer similarly than with numbered citations
% (It has been said that blaa blaa~\citep{blaablaa}).
%\usepackage[square]{natbib}

% The alltt package provides an all-teletype environment that acts
% like verbatim but you can use LaTeX commands in it. Uncomment if
% you want to use this environment.
% \usepackage{alltt}

% The eurosym package provides a euro symbol. Use with \euro{}
\usepackage{eurosym}

% Verbatim provides a standard teletype environment that renderes
% the text exactly as written in the tex file. Useful for code
% snippets (although you can also use the listings package to get
% automatic code formatting).
\usepackage{verbatim}
\usepackage{float}

% The listing package provides automatic code formatting utilities
% so that you can copy-paste code examples and have them rendered
% nicely. See the package documentation for details.
% \usepackage{listings}

% The fancuvrb package provides fancier verbatim environments
% (you can, for example, put borders around the verbatim text area
% and so on). See package for details.
% \usepackage{fancyvrb}

% Supertabular provides a tabular environment that can span multiple
% pages.
%\usepackage{supertabular}
% Longtable provides a tabular environment that can span multiple
% pages. This is used in the example acronyms file.
\usepackage{longtable}

% The fancyhdr package allows you to set your the page headers
% manually, and allows you to add separator lines and so on.
% Check the package documentation.
% \usepackage{fancyhdr}

% Subfigure package allows you to use subfigures (i.e. many subfigures
% within one figure environment). These can have different labels and
% they are numbered automatically. Check the package documentation.
\usepackage{subfigure}

% The titlesec package can be used to alter the look of the titles
% of sections, chapters, and so on. This example uses the ``medium''
% package option which sets the titles to a medium size, making them
% a bit smaller than what is the default. You can fine-tune the
% title fonts and sizes by using the package options. See the package
% documentation.
\usepackage[medium]{titlesec}

% The TikZ package allows you to create professional technical figures.
% The learning curve is quite steep, but it is definitely worth it if
% you wish to have really good-looking technical figures.
\usepackage{tikz}
% You also need to specify which TikZ libraries you use
\usetikzlibrary{positioning}
\usetikzlibrary{calc}
\usetikzlibrary{arrows}
\usetikzlibrary{decorations.pathmorphing,decorations.markings}
\usetikzlibrary{shapes}
\usetikzlibrary{patterns}


% The aalto-thesis package provides typesetting instructions for the
% standard master's thesis parts (abstracts, front page, and so on)
% Load this package second-to-last, just before the hyperref package.
% Options that you can use:
%   mydraft - renders the thesis in draft mode.
%             Do not use for the final version.
%   doublenumbering - [optional] number the first pages of the thesis
%                     with roman numerals (i, ii, iii, ...); and start
%                     arabic numbering (1, 2, 3, ...) only on the
%                     first page of the first chapter
%   twoinstructors  - changes the title of instructors to plural form
%   twosupervisors  - changes the title of supervisors to plural form
%\usepackage[twosupervisors]{aalto-thesis}
%\usepackage[mydraft,doublenumbering]{aalto-thesis}
\usepackage[mydraft]{aalto-thesis}
\usepackage{pdfpages}

% Hyperref
% ------------------------------------------------------------------
% Hyperref creates links from URLs, for references, and creates a
% TOC in the PDF file.
% This package must be the last one you include, because it has
% compatibility issues with many other packages and it fixes
% those issues when it is loaded.
\RequirePackage[pdftex]{hyperref}
% Setup hyperref so that links are clickable but do not look
% different
\hypersetup{colorlinks=false,raiselinks=false,breaklinks=true}
\hypersetup{pdfborder={0 0 0}}
\hypersetup{bookmarksnumbered=true}
% The following line suggests the PDF reader that it should show the
% first level of bookmarks opened in the hierarchical bookmark view.
\hypersetup{bookmarksopen=true,bookmarksopenlevel=1}
% Hyperref can also set up the PDF metadata fields. These are
% set a bit later on, after the thesis setup.


% Thesis setup
% ==================================================================
% Change these to fit your own thesis.
% \COMMAND always refers to the English version;
% \FCOMMAND refers to the Finnish version; and
% \SCOMMAND refers to the Swedish version.
% You may comment/remove those language variants that you do not use
% (but then you must not include the abstracts for that language)
% ------------------------------------------------------------------
% If you do not find the command for a text that is shown in the cover page or
% in the abstract texts, check the aalto-thesis.sty file and locate the text
% from there.
% All the texts are configured in language-specific blocks (lots of commands
% that look like this: \renewcommand{\ATCITY}{Espoo}.
% You can just fix the texts there. Just remember to check all the language
% variants you use (they are all there in the same place).
% ------------------------------------------------------------------
\newcommand{\TITLE}{In-house software development process:}
\newcommand{\SUBTITLE}{A User-centered Approach}
\newcommand{\FTITLE}{Yrityksen sisäinen ohjelmistokehitysprosessi:}
\newcommand{\FSUBTITLE}{Käyttäjäkeskeinen lähestymistapa}
%\newcommand{\STITLE}{Den stora stygga vargen:}
%\newcommand{\SUBTITLE}{Subtitle}
%\newcommand{\FSUBTITLE}{Alaotsikko}
%\newcommand{\SSUBTITLE}{Lilla Vargens universum}
\newcommand{\DATE}{June 5, 2013}
\newcommand{\FDATE}{5. kesäkuuta 2013}
%\newcommand{\SDATE}{Den 18 Juni 2011}

% Supervisors and instructors
% ------------------------------------------------------------------
% If you have two supervisors, write both names here, separate them with a
% double-backslash (see below for an example)
% Also remember to add the package option ``twosupervisors'' or
% ``twoinstructors'' to the aalto-thesis package so that the titles are in
% plural.
% Example of one supervisor:
\newcommand{\SUPERVISOR}{Professor Marko Nieminen}
\newcommand{\FSUPERVISOR}{Professori Marko Nieminen}
%\newcommand{\SSUPERVISOR}{Professor Antti Ylä-Jääski}
% Example of twosupervisors:
%\newcommand{\SUPERVISOR}{Professor Marko Nieminen\\
%  Professor Marko Nieminen}
%\newcommand{\FSUPERVISOR}{Professori Marko Nieminen\\
%  Professori Marko Nieminen}
%\newcommand{\SSUPERVISOR}{Professor Antti Ylä-Jääski\\
%  Professor Pekka Perustieteilijä}

% If you have only one instructor, just write one name here
\newcommand{\INSTRUCTOR}{Jouni Kuusinen M.Sc. (Tech.)}
\newcommand{\FINSTRUCTOR}{Filosofian maisteri Jouni Kuusinen}
%\newcommand{\SINSTRUCTOR}{Diplomingenjör Olli Ohjaaja}
% If you have two instructors, separate them with \\ to create linefeeds
% \newcommand{\INSTRUCTOR}{Olli Ohjaaja M.Sc. (Tech.)\\
%  Elli Opas M.Sc. (Tech)}
%\newcommand{\FINSTRUCTOR}{Diplomi-insinööri Olli Ohjaaja\\
%  Diplomi-insinööri Elli Opas}
%\newcommand{\SINSTRUCTOR}{Diplomingenjör Olli Ohjaaja\\
%  Diplomingenjör Elli Opas}

% If you have two supervisors, it is common to write the schools
% of the supervisors in the cover page. If the following command is defined,
% then the supervisor names shown here are printed in the cover page. Otherwise,
% the supervisor names defined above are used.
%\newcommand{\COVERSUPERVISOR}{Professor Antti Ylä-Jääski, Aalto University\\
%  Professor Pekka Perustieteilijä, University of Helsinki}

% The same option is for the instructors, if you have multiple instructors.
% \newcommand{\COVERINSTRUCTOR}{Olli Ohjaaja M.Sc. (Tech.), Aalto University\\
%  Elli Opas M.Sc. (Tech), Aalto SCI}


% Other stuff
% ------------------------------------------------------------------
\newcommand{\PROFESSORSHIP}{Usability and User Interfaces}
\newcommand{\FPROFESSORSHIP}{Käytettävyys ja käyttöliittymät}
%\newcommand{\SPROFESSORSHIP}{Datakommunikationsprogram}
% Professorship code is the same in all languages
\newcommand{\PROFCODE}{T-121}
\newcommand{\KEYWORDS}{Usability, ERP, Software Development Process, Process Measurement, Cognitive walkthrough, Remote Usability Evaluation, SUS, Contextual Inquiry, ISI}
\newcommand{\FKEYWORDS}{Käytettävyys, ERP, Ohjelmistokehitysprosessi, Prosessimittaus, Kognitiivinen läpikäynti, Käytettävyyden etäarviointi, SUS, Kontekstuaalinen tutkimus, ISI}
%\newcommand{\SKEYWORDS}{omsättning, kassaflöde, värdepappersmarknadslagen,
%yrkesutövare, intresseföretag, verifieringskedja}
\newcommand{\LANGUAGE}{English}
\newcommand{\FLANGUAGE}{Englanti}
%\newcommand{\SLANGUAGE}{Engelska}

% Author is the same for all languages
\newcommand{\AUTHOR}{Antti Paananen}


% Currently the English versions are used for the PDF file metadata
% Set the PDF title
\hypersetup{pdftitle={\TITLE\ \SUBTITLE}}
% Set the PDF author
\hypersetup{pdfauthor={\AUTHOR}}
% Set the PDF keywords
\hypersetup{pdfkeywords={\KEYWORDS}}
% Set the PDF subject
\hypersetup{pdfsubject={Master's Thesis}}


% Layout settings
% ------------------------------------------------------------------

% When you write in English, you should use the standard LaTeX
% paragraph formatting: paragraphs are indented, and there is no
% space between paragraphs.
% When writing in Finnish, we often use no indentation in the
% beginning of the paragraph, and there is some space between the
% paragraphs.

% If you write your thesis Finnish, uncomment these lines; if
% you write in English, leave these lines commented!
% \setlength{\parindent}{0pt}
% \setlength{\parskip}{1ex}

% Use this to control how much space there is between each line of text.
% 1 is normal (no extra space), 1.3 is about one-half more space, and
% 1.6 is about double line spacing.
% \linespread{1} % This is the default
% \linespread{1.3}

% Bibliography style
% acm style gives you a basic reference style. It works only with numbered
% references.
\bibliographystyle{acm}
% Plainnat is a plain style that works with both numbered and name citations.
%\bibliographystyle{plainnat}


% Extra hyphenation settings
% ------------------------------------------------------------------
% You can list here all the files that are not hyphenated correctly.
% You can provide many \hyphenation commands and/or separate each word
% with a space inside a single command. Put hyphens in the places where
% a word can be hyphenated.
% Note that (by default) LaTeX will not hyphenate words that already
% have a hyphen in them (for example, if you write ``structure-modification
% operation'', the word structure-modification will never be hyphenated).
% You need a special package to hyphenate those words.
\hyphenation{di-gi-taa-li-sta yksi-suun-tai-sta ana-ly-sis pro-duct}



% The preamble ends here, and the document begins.
% Place all formatting commands and such before this line.
% ------------------------------------------------------------------
\begin{document}
% This command adds a PDF bookmark to the cover page. You may leave
% it out if you don't like it...
\pdfbookmark[0]{Cover page}{bookmark.0.cover}
% This command is defined in aalto-thesis.sty. It controls the page
% numbering based on whether the doublenumbering option is specified
\startcoverpage

% Cover page
% ------------------------------------------------------------------
% Options: finnish, english, and swedish
% These control in which language the cover-page information is shown
\coverpage{english}


% Abstracts
% ------------------------------------------------------------------
% Include an abstract in the language that the thesis is written in,
% and if your native language is Finnish or Swedish, one in that language.

% Abstract in English
% ------------------------------------------------------------------
\thesisabstract{english}{
-
%\fixme{Abstract text goes here (and this is an example how to use fixme).}
%Fixme is a command that helps you identify parts of your thesis that still
%require some work. When compiled in the custom \texttt{mydraft} mode, text
%parts tagged with fixmes are shown in bold and with fixme tags around them. When
%compiled in normal mode, the fixme-tagged text is shown normally (without
%special formatting). The draft mode also causes the ``Draft'' text to appear on
%the front page, alongside with the document compilation date. The custom
%\texttt{mydraft} mode is selected by the \texttt{mydraft} option given for the
%package \texttt{aalto-thesis}, near the top of the \texttt{thesis-example.tex}
%file.
%The thesis example file (\texttt{thesis-example.tex}), all the chapter content
%files (\texttt{1introduction.tex} and so on), and the Aalto style file
%(\texttt{aalto-thesis.sty}) are commented with explanations on how the Aalto
%thesis works. The files also contain some examples on how to customize various
%details of the thesis layout, and of course the example text works as an
%example in itself. Please read the comments and the example text; that should
%get you well on your way!
}

% Abstract in Finnish
% ------------------------------------------------------------------
\thesisabstract{finnish}{
-
}

% Abstract in Swedish
% ------------------------------------------------------------------
%\thesisabstract{swedish}{
%Lilla Vargens universum är det tredje fiktiva universumet inom huvudfäran av de
%tecknade disneyserierna - de övriga två är Kalle Ankas och Musse Piggs
%universum. Figurerna runt Lilla Vargen kommer huvudsakligen frän tre källor ---
%dels persongalleriet i kortfilmen Tre smä grisar frän 1933 och dess uppföljare,
%dels längfilmen Sängen om Södern frän 1946, och dels frän episoden ``Bongo'' i
%längfilmen Pank och fägelfri frän 1947. Framför allt de två första har
%sedermera även kommit att leva vidare, utvidgas och införlivas i varandra genom
%tecknade serier, främst sädana producerade av Western Publishing för
%amerikanska Disneytidningar under ären 1945--1984.

%Världen runt Lilla Vargen är, i jämförelse med den runt Kalle Anka eller Musse
%Pigg, inte helt enhetlig, vilket bland annat märks i Bror Björns skiftande
%personlighet. Den har även varit betydligt mer öppen för influenser frän andra
%Disneyvärldar, inte minst de tecknade längfilmerna. Ytterligare en skillnad är
%att varelserna i vargserierna förefaller stä närmare sina förebilder inom den
%verkliga djurvärlden. Att vargen Zeke vill äta upp grisen Bror Duktig är till
%exempel ett ständigt äterkommande tema, men om katten Svarte Petter skulle fä
%för sig att äta upp musen Musse Pigg skulle detta antagligen höja ett och annat
%ögonbryn bland läsarna.}


% Acknowledgements
% ------------------------------------------------------------------
% Select the language you use in your acknowledgements
\selectlanguage{english}

% Uncomment this line if you wish acknoledgements to appear in the
% table of contents
%\addcontentsline{toc}{chapter}{Acknowledgements}

% The star means that the chapter isn't numbered and does not
% show up in the TOC
\chapter*{Acknowledgements}


-

-
\vskip 10mm

\noindent Espoo, \DATE
\vskip 5mm
\noindent\AUTHOR

% Acronyms
% ------------------------------------------------------------------
% Use \cleardoublepage so that IF two-sided printing is used
% (which is not often for masters theses), then the pages will still
% start correctly on the right-hand side.
\cleardoublepage
% Example acronyms are placed in a separate file, acronyms.tex
% \input{acronyms}

\addcontentsline{toc}{chapter}{Abbreviations and Acronyms}
\chapter*{Abbreviations}

% The longtable environment should break the table properly to multiple pages,
% if needed

\noindent
\begin{longtable}{@{}p{0.25\textwidth}p{0.7\textwidth}@{}}
CI & Contextual Inquiry \\
CW & Cognitive Walkthrough\\
ERP & Enterprise Resource Planning \\
HCI & Human-Computer Interaction \\
ISI & Interaction Sequence Illustration \\
IT & Information Technology \\
SUS & System Usability Scale \\
UI & User Interface \\
UX & User Experience \\


%FLUTE  & The File Delivery over Unidirectional Transport protocol \\
%e.g.& for example (do not list here this kind of common acronymbs or abbreviations, but only those that are essential for understanding the content of your thesis. \\
%note & Note also, that this list is not compulsory, and should be omitted if you have only few abbreviations

\end{longtable}


% Table of contents
% ------------------------------------------------------------------
\cleardoublepage
% This command adds a PDF bookmark that links to the contents.
% You can use \addcontentsline{} as well, but that also adds contents
% entry to the table of contents, which is kind of redundant.
% The text ``Contents'' is shown in the PDF bookmark.
\pdfbookmark[0]{Contents}{bookmark.0.contents}
\tableofcontents

% List of tables
% ------------------------------------------------------------------
% You only need a list of tables for your thesis if you have very
% many tables. If you do, uncomment the following two lines.
% \cleardoublepage
% \listoftables

% Table of figures
% ------------------------------------------------------------------
% You only need a list of figures for your thesis if you have very
% many figures. If you do, uncomment the following two lines.
% \cleardoublepage
% \listoffigures

% The following label is used for counting the prelude pages
\label{pages-prelude}
\cleardoublepage

%%%%%%%%%%%%%%%%% The main content starts here %%%%%%%%%%%%%%%%%%%%%
% ------------------------------------------------------------------
% This command is defined in aalto-thesis.sty. It controls the page
% numbering based on whether the doublenumbering option is specified
\startfirstchapter

% Add headings to pages (the chapter title is shown)
\pagestyle{headings}

% The contents of the thesis are separated to their own files.
% Edit the content in these files, rename them as necessary.
% ------------------------------------------------------------------

% \input{1introduction.tex}

\chapter{Introduction}
\label{chapter:introduction}
In this chapter, the background and reasoning for the thesis is described together with the focus and limitations of the research. In the text, research problems and the structure of the thesis will be also defined. 

\section{Motivation and aim of the thesis}
\label{sec:motivationandaim}
In the 1980s, when the usage of personal computers (PCs) became more common, software design practices were still falsely assuming that the users were knowledgeable and competent in computer science. As an outcome, big part of the users were practically incapable of using operating systems and applications.
During these times, the concepts of Human Computer Interaction (HCI) and usability, became important. Since then, the design process of interactive software for common people emphasized usability. This process is called human-centered design. \cite{RefWorks:9}

The term Enterprise Resource Planning (ERP) was invented in the early 1990s.\cite{RefWorks:3} The purpose of the ERP software is to offer techniques and concepts for integrated and thorough management of business, as well as making it more efficient.
The usage of ERP software has increased globally and nowadays even service organizations have invested a lot of resources in ERP implementation.\cite{RefWorks:1, RefWorks:7} 

Despite the importance of the efficiency aspect, the usability of ERP systems is not a widely studied topic. However, weaknesses in usability may lead into a low productivity and make it harder for users to achieve their goals.\cite{RefWorks:2} 

The aim of this thesis is to examine how the usability of a service-oriented ERP system can be enhanced by integrating usability inquiries, inspections and measures into the software development process. In the research, one well defined business process is examined and the state of its usability in the system is determined  by using variety of applicable methods:
\begin{itemize}
\item Contextual Inquiry to define the business process.
\item Cognitive walkthrough for usability inspection.
\item Interaction Sequence Illustration (ISI) to measure the amount of interaction steps and to understand them.
\item System Usability Scale (SUS) to give a global view of subjective assessments of usability.
\item Remote Usability evaluation and Usability logging for remote usability evaluation.
\end{itemize}
The measurements are focused on time, error rate and user satisfaction.
\\
\\
\indent According to the ISO standard of Human-centered design for interactive systems \cite{RefWorks:16}, many benefits can be gained by using human-centered methods as a part of software development process. The productivity of an individual user can be increased, as well as the operational efficiency of an organization. Usable and useful systems also reduce training and help-desk costs together with stress and discomfort since they are understandable by the users. In other words, human-centered design improves the UX (User Experience).

The benefits of human-centered methods (for a software development process) are increased total lifecycle of product and likelihood of project succeeding on time and within budget. Human-centric approach also decrease the risk of software being rejected by the users or failing to meet the requirements. \cite{RefWorks:16}

\section{Background and research questions}
\label{sec:backgroundandresearchquestions}
The subscriber of this thesis is a middle-sized company which is offering information services globally and practicing in-house software development. Because of the fast pace of growth, the company is willing to reform their current ERP system as well as the whole software development process. The aim of this thesis is to join usability perspective into this process and give answers to following research questions.

\begin{itemize}
\item \textbf{\emph{How usability methods can help to identify critical disparities in the usage of a system?}}

Understanding the differences in the system usage between individuals can help to understand and deploy best practices throughout the organization and therefore improve efficiency.

\item \textbf{\emph{How the user efficiency is affected by the usability measurements?}}

It is important to find the most effective and usable user interface solutions and thus decrease the average time spent on tasks. Local differences can be tracked with remote usability measurements.

\item \textbf{\emph{What usability methods can be practically joined with the software development process of an ERP system?}}

Finding practical and efficient usability methods to be joined with the software development process can improve the quality of the end product.

\end{itemize}



%Usability perspective of the software development process is being considered because of many reasons. There has %been noticeable differences between country offices using the same system and working with the same processes. %Because of the growth of the company, the system must be more efficient, but also pleasant to use.
%\begin{itemize}
%\item Efficiency differences betw

%\end{itemize}

%\begin{itemize}
%                    \item information services
%                    \item publication delivery
%                    \item database services
%                \end{itemize}
%            \item general info about the company
%                \begin{itemize}
%                    \item middle-sized, ~200 employees
%                    \item fast pace of growth
%                    \item personnel service
%                \end{itemize}
%            \item what will be done
%                \begin{itemize}
%                    \item SOA 
%                    \item new process model for software development
%                \end{itemize}
%        \end{itemize}
%    
%    \section{Research problems}
%    \label{sec:researchproblems}
%        \begin{itemize}
%            \item Efficiency differences in country offices.
%            \item Need for more efficient ERP.
%            \item Usability issues
%            \item Can usability methods benefit the understanding of business process
%        \end{itemize}
%    
\section{Scope and structure of the thesis}
\label{sec:thescopeandstructureofthethesis}

This thesis covers research about usability of the in-house software development process and its scope does not include any other aspects of the process. The literature  research consists of a few usability methods and even though the target of the research is ERP software, literature about them are not covered in the thesis. The results of the research may not be suitable for every organization.

The first actual chapter of the thesis is about the usability methods. Every usability method used in the research is discussed carefully. In the second chapter the process experiment is being introduced. It covers the experiment steps and the implementation details. In the third chapter, the data gathered in the experiment, and the implementation process is being analyzed. In the last chapter the research will be summed up and discussed.

    
    
%\chapter{Background}
%\label{chapter:background}

%Transitions mentioned in Section~\ref{section:structure} are used also




%\subsection{Finding sources}


%\begin{itemize}
% You can use this command to set the items in the list closer to each other
% (ITEM SEParation, the vertical space between the list items)
%\setlength{\itemsep}{0pt}
%\item Nelli Portal (Aalto Library): \url{http://www.nelliportaali.fi}
%\item ACM Digital library: \url{http://portal.acm.org/}
%\item IEEExplore: \url{http://ieeexplore.ieee.org}
%\item ScienceDirect: \url{http://www.sciencedirect.com/}
%\item \ldots although Google Scholar (\url{http://scholar.google.com/}) will
%find links to most of the articles from the abovementioned sources, if you
%search from within the university network
%\end{itemize}


%information~\cite{howfindinfo}.
%(\url{http://scholar.google.fi/}). It finds academic publications whereas
%\subsection{Referring to sources}
%instructions for many styles~\cite{bibinstructions}. You should ask
%give short examples that are marked with \emph{emphasised text}.
%\emph{Haapasalo~\cite{HaapasaloThesis} researched database algorithms


%If your paragraph has several sources, the above mentioned styles are
%not proper. The reader of your thesis cannot know which of your
%sources give which of the statements. In this case, it is better to
%use more finegraded refering where the reference markings that are
%embedded in the sentences. For example, \emph{the multiversion B+-tree
%  (MVBT) index of Becker et al.~\cite{becker:1996:mvbt} allows database
%  users to query old versions of the database, but the index is not
%  transactional.
%  It's successor, the transactional MBVT (TMVBT), allows a single transaction
%  running in its own thread or process to update the database concurrently
%  with other transactions that only read the
%  database~\cite{haapasalo:2009:tmvbt}.
%  Further development, titled the concurrent MBVT (CMVBT),
%  allows several transactions to perform updates to the database at the same
%  time~\cite{HaapasaloThesis}}.
%  Here, the references are marked before
%  the period in the sentences where they are used.

%  X~[\ldots] according to ms Y~[\ldots] defined that}, if you find a
%\textbf{not} to do it: \emph{\cite{HaapasaloThesis} describes
% \input{3environment.tex}

\chapter{Methods}
\label{chapter:methods}
In order to be able to discover reliable research data, the research methods must be understood thoroughly. In this research, the data is gathered with a few types of usability methods and they are selected according to their practicality and utility.
Inquiries are used to study the business process and the process itself is measured from many different aspects and also remote evaluation is utilized to gather data easily from distributed locations. 

Methods are used in different stages of the research to fulfill the needs of human-centered design activities and they are described in detail in the following chapter.  

\section{Automated Remote Usability Evaluation}
\label{sec:rue}
 
    

\section{Contextual Inquiry}
\label{sec:cinquiry}
\texttt{TÄHÄN VIELÄ SEMISTRUCTURED STUFF!!!}
Contextual Inquiry was originally developed to meet three requirements. Firstly it was supposed to identify a design process for systems that will be used similarly in different business contexts and in different cultures. It was also supposed to indentify a convenient process for gathering user information in limited time and finally to identify a way to acquire information about users' work in eligible format. In addition to those requirements, the technique was noticed to be capable of much more. CI cherish participatory design and, because of that quality, users are able to involve in the design. Users' contributes to the design by providing a deep understanding about the nature of the work. This is done through inquiry and it's a basis for fundamental work concepts. \cite{RefWorks:14} 

\subsection{Fundamentals of Contextual Inquiry}
Contextual Inquiry can be said to be an apprenticeship compressed in time. The basis of the method is premised on the idea of user being the expert instead of the interviewer, but unlike an apprentice, the interviewer neither learns the work by doing it, nor has the same amount of time available for learning. \cite{RefWorks:21} CI differs from the traditional master-apprenticeship model in other ways also. A few fundamental principles of Contextual Inquiry are said to be essential in order to meet the specific needs of design problems \cite{RefWorks:21, Refworks:22}.  These principles are understanding the context of the work, creating a partnership, interpreting the work and steering the focus during the interview. \cite{RefWorks:21}

Understanding the \textbf{context} of the work is the baseline of Contextual Inquiry. To gain the understanding about the work structure, the interviewer must pursue understanding about the details of users' work and these details can be found by following the users' actions at work. In general, it is important that interviewer avoids gathering abstract or summarized information about the context.\cite{RefWorks:21}

It is essential to create collaborative environment and a \textbf{partnership} between the user and the interviewer while the real life work structure and activities are tried to be understood thoroughly. Partnership is an equal relationship between the interviewer and the user. In comparison to traditional interview or master-apprentice approaches, partnership doesn't give any power advantage to either parties. Instead, it fosters the interviewer's expertise to see the work structure and the user's expertise to do the work. There are many advantages in partnership approach. For example by paying attention to details and structure of work, interviewer can also teach the user to attend to them. In the best case scenario the interviewer and the user watch the work structure and think about design possibilities together. In this kind of scenario it is common that the work is suspended, while the parties discuss about the work structure, and then return again. For interviewer asking feedback for design ideas is also engouraged.\cite{RefWorks:21}




 

As soon as the understanding about the work is available design for a system model can be created.

\subsection{Contextual Inquiry in practice}
-go to workplace


\section{Cognitive walkthrough}
\label{sec:cognitivewalkthrough}

Cognitive Walkthrough (CW) was developed in the early nineties and it was originally intended to help reviewing 'walk-up-and-use' interfaces, such as automatic teller machine (ATM). CW is a formal usability inspection method for professionals, involved in development process. The key concept behind CW is to use theory as a guide for design review. It is easy to understand and apply and therefore feasible to use in a regular development process. \cite{RefWorks:19, RefWorks:18} 	

\section{Interaction Sequence Illustration}
\label{sec:isi}
Considering the practical impacts of the thesis, it is important to use methods and measures which can shore up the software development process in a real-use context. This is why the process and it's interactions are reviewed using ISI method, which utilizes 				authentic use context and real users. \cite{RefWorks:17} ISI was originally developed for evaluating the usability of IT tasks carried out in healthcare industry, but there are no defined reasons why it could not be utilized in different environments.

\subsection{Description of the method}
Interaction Sequence Illustration is a low level analysis method for human-computer interaction. It uses data acquired during the contextual inquiry process and doesn't gather any of its own. However, the inquiry data has to be complemented with documentation 			about 	interaction activities. There are three objectives for ISI method to handle. The first is to demonstrate how the user perceives the system. The second one is to identify and document activities, and to discover problems. The first two objectives creates the 			third one, which is to support the user-centered design and development. \cite{RefWorks:17}  The strength of ISI lies in the analysis of data and it can be used to compare the interaction sequences of two of or more UI implementations. On the other hand it can 			provide prominent information on only one UI's interaction sequence. 

\subsection{Utilization of Interaction Sequence Illustration}
The utilization of Interaction Sequence Illustration can be simplified in seven steps (see Figure \ref{fig:isi_chart}). First the data need to be collect (notes and screenshots) alongside Contextual Inquiry interview.
Then the screenshots need to be arranged in the right order and all the superfluous data need to be removed. The third phase is to count the interaction steps based on screenshots and activity analysis.
After the analysis the screenshots needs to be modified and important details highlighted. Then the sequence numbers as well as a  detailed description text should be added to give a profound understanding about the actions. The outcomes of the method are 			illustration, or illustrations depending on the number of research objectives, of the interaction stages. Every stage contains step-by-step illustration (or illustrations) of user-computer interaction. \cite{RefWorks:17}
	
\begin{figure}[H]
  	\centering
  	\includegraphics[width=1.1\textwidth]{./images/isi_chart.png}
  	\caption{Interaction Sequence Illustration steps and outcomes.}
	\label{fig:isi_chart}
\end{figure}

\section{System Usability Scale}
\label{sec:sus}
In his paper Brooke \cite{RefWorks:10} argued that usability is not any real existing quality, but a good usability artifact is \emph{appropriate to its purpose}. In other words "the usability of any tool or system has to be viewed in terms of the context in which it is 			used, and its appropriateness to that context"\cite{RefWorks:10}. Still in many cases, context related usability evaluation is not desirable. The reason for this is that a large scale context analysis is usually neither cost-efficient nor practical.\cite{RefWorks:10} 
SUS responds to these challenges by offering an easy and quick way to get subjective ratings about the usability of a system. It is not limited to any specific technology, which makes it universal tool for usability evaluation. \cite{RefWorks:12}

\subsection{Description of the System Usability Scale}
System Usability Scale (see \nameref{app:susform}) is a ten-item \emph{Likert scale}, meaning that every item consist of the scale of five, ranging from "Strongly disagree" to "Strongly agree". The questionnaire is generally being filled right after the possibility to use 		the system to be evaluated. The focus should be on immediate responses and too much time shouldn't be given to the respondents. \cite{RefWorks:10} 
\texttt{TÄHÄN LISÄÄ!!! TEN YEARS OF SUS -ARTIKKELISTA}


\subsection{System Usability Scale in practice}
The outcome of SUS is a single value which express the overall usability of the system. The value consist of all the items and none of them are meaningful as such. System Usability Scale can be calculated by first summing the score contribution (range from 0 to 4) 			from each item. Before summing the scale positions of the items 1,3,5,7 and 9 need to be subtracted by one and the scale positions of the items 2,4,6,8 and 10 need to be subtracted from 5. The last step is to multiply the sum of the scores by 2.5 to get the overall 			SUS value, which will range from minimum of 0 to maximum of 100. \cite{RefWorks:10} The resulting single score is an easy-to-understand measure, and can therefore be discussed with the wide range of stakeholders. \cite{RefWorks:12} 

	

\section{Other measures}
\label{sec:othermeasures}
The last process measurements used in the research are simply the time which was consumed while carrying out the task and the success rate of the task.


 
    
\chapter{Process experiment}
\label{processexperiment}

Human-centered design consist a few activities and iterative process (see Figure \ref{fig:hci_process}). \cite{RefWorks:16} The empirical part of this thesis adapts human-centered desing principles and the steps of the process experiment are highly linked to its activities. The steps are described in detail in section \ref{sec:steps}. This chapter will also describe the implementation phases of the experiment. 

\texttt{TÄHÄN JOTAIN CONTEXTUAL DESIGNISTÄ!!!!!}

\begin{figure}[H]
  	\centering
    	\includegraphics[width=1.0\textwidth]{./images/hci_process.png}
  	\caption{Human-centered design activities.\cite{RefWorks:16}}
	\label{fig:hci_process}
\end{figure}

    \section{Steps}
    \label{sec:steps}
        \begin{itemize}
            \item Creating the model for gathering data
            \item Modified contextual inquiry 
            \item Process measurement methods
            \item Analysis 1.
            \item Prototype creation
            \item Remote evaluation - process measurement methods.
            \item Analysis 2.
        \end{itemize}

    \section{Implementation}
    \label{sec:implementation}
	\texttt{Kuinka monta osallistujaa, millainen ympäristö, missä maassa, minkä ikäisiä osallistujat, kauanko tehneet työtä}
    
\chapter{Analysis}
\label{chapter:analysis}

    \section{Results}
    \label{sec:results}
    -Comparison between country offices
    -Comparison between individuals (esim. kuinka kauan kesti tietyn toiminnon tekeminen) / Overall comparison (esim. kaikkien koehenkilöiden yhteinen kehitys.)
- Interaction sequence illustration (esim. kuinka monta steppiä ennen ja jälkeen)

    \section{Implementation analysis}
    \label{sec:implementationanalysis}
    -Should these methods be implemented as a part of the process or not.

%When you use \texttt{pdflatex} to render your thesis, you can include PDF images
%directly, as shown by Figure~\ref{fig:indica_model} below.

%\begin{figure}[ht]
%  \begin{center}
%    \includegraphics[width=\textwidth]{example_indica_model.pdf}
%    \caption{The INDICA two-layered value chain model.}
%    \label{fig:indica_model}
%  \end{center}
%\end{figure}

%You can also include JPEG or PNG files, as shown by Figure~\ref{fig:eeyore}.

%\begin{figure}[ht]
%  \begin{center}
%    \includegraphics[width=9cm]{example_ihaa.jpg}
%    \caption{Eeyore, or Ihaa, a very sad donkey.}
%    \label{fig:eeyore}
%  \end{center}
%\end{figure}


%If you have PS or EPS files, you can use the tools \texttt{ps2pdf} or
%\texttt{epspdf} to convert your PS and EPS files to PDF\@.

% Comment: If your sentence ends in a capital letter, like here, you should
% write \@ before the period; otherwise LaTeX will assume that this is not
% really an end of the sentence and will not put a large enough space after the
% period. That is, LaTeX assumes that you are (for example), enumerating using
% capital roman numerals, like I. do something, II. do something else. In this
% case, the periods do not end the sentence.

% Similarly, if you do need a normal space after a period (instead of
% the longer sentence separator), use \  (backslash and space) after the
% period. Like so: a.\ first item, b.\ second item.


%WYSIWYG vector editor that allows you to save directly to PDF\@.

%Excel to PDF format, and then add them; or use \texttt{gnuplot}, which can
%something like \texttt{set term pdf \ldots}.


%rather steep learning curve. Locate the manual (\texttt{pgfmanual.pdf}) from
%graphics is shown in Figure~\ref{fig:page-merge}.

%\begin{figure}[ht]
%  \begin{center}
%    \input{example_page-merge.tex}
%    \caption{Example of a multiversion database page merge. This figure has
%    been taken from the PhD thesis of Haapasalo~\cite{HaapasaloThesis}.}
%    \label{fig:page-merge}
%  \end{center}
%\end{figure}


% These definitions are only used in the example images; you will not
% need them for your thesis...
%\newlength{\graphdotsize}
%\setlength{\graphdotsize}{1.7pt}
%\newlength{\graphgridsize}
%\setlength{\graphgridsize}{1.2em}
%\begin{figure}[ht]
%\begin{center}
%\subfigure[Examples of obstruction graphs for the Ferry Problem]{
%  \input{example_obstruction-grouped.tex}
%}
%\subfigure[Examples of star graphs]{
%  \input{example_general-star-graphs.tex}
%}
%\caption{Examples of graphs draw with TikZ. These figures have been taken from a
%course report for the graph theory course~\cite{FerryProblem}.}
%\label{fig:tikz-examples}
%\end{center}
%\end{figure}



% \input{4methods.tex}


% An example of a traditional LaTeX table
% ------------------------------------------------------------------
% A note on underfull/overfull table cells and tables:
% ------------------------------------------------------------------
% In professional typography, the width of the text in a page is always a lot
% less than the width of the page. If you are accustomed to the (too wide) text
% areas used in Microsoft Word's standard documents, the width of the text in
% this thesis layout may suprise you. However, text in a book needs wide
% margins. Narrow text is easier to read and looks nicer. Longer lines are
% hard to read, because the start of the next line is harder to locate when
% moving from line to the next.
% However, tables that are in the middle of the text often would require a wider
% area. By default, LaTeX will complain if you create too wide tables with
% ``overfull'' error messages, and the table will not be positioned properly
% (not centered). If at all possible, try to make the table narrow enough so
% that it fits to the same space as the text (total width = \textwidth).
% If you do need more space, you can either
% 1) ignore the LaTeX warnings
% 2) use the textpos-package to manually position the table (read the package
%    documentation)
% 3) if you have the table as a PDF document (of correct size, A4), you can use
%    the pdfpages package to include the page. This overrides the margin
%    settings for this page and LaTeX will not complain.
% ------------------------------------------------------------------
% Another note:
% ------------------------------------------------------------------
% If your table fits to \textwidth, but the cells are so narrow that the text
% in p{..}-formatted cells does not flow nicely (you get underfull warnings
% because LaTeX tries to justify the text in the cells) you can manually set
% the text to unjustified by using the \raggedright command for each cell
% that you do not want to be justified (see the example below). \raggedleft
% is also possible, of course...
% ------------------------------------------------------------------
% If you need to have linefeeds (\\) inside a cell, you must create a new
% paragraph-formatting environment inside the cell. Most common ones are
% the minipage-environment and the \parbox command (see LaTeX documentation
% for details; or just google for ``LaTeX minipage'' and ``LaTeX parbox'').
%\begin{table}
%\begin{tabular}{|p{2cm}|p{3.8cm}|p{4.5cm}|p{1.1cm}|}
% Alignment of sells: l=left, c=center, r=right.
% If you want wrapping lines, use p{width} exact cell widths.
% If you want vertical lines between columns, write | above between the letters
% Horizontal lines are generated with the \hline command:
%\hline % The line on top of the table
%\textbf{Code} & \textbf{Name} & \textbf{Methods} & \textbf{Area} \\
%\hline
% Place a & between the columns
% In the end of the line, use two backslashes \\ to break the line,
% then place a \hline to make a horizontal line below the row

%    Software & \raggedright Computer simulations, mathematical modeling,

%\hline
%\multicolumn{2}{|p{6.25cm}|}{Mat-2.3170 Simulation (here is an example of
% multicolumn for tables)}& Details of how to build simulations & T-110 \\
% The multicolumn command takes the following 3 arguments:
% the number of cells to merge, the cell formatting for the new cell, and the
% contents of the cell
%\hline
%S-38.3184 & Network Traffic Measurements and Analysis
%& \raggedright How to measure and analyse network
%  traffic & T-110 \\ \hline
%\end{tabular} % for really simple tables, you can just use tabular
% You can place the caption either below (like here) or above the table
%\caption{Research methodology courses}
% Place the label just after the caption to make the link work
%\label{table:courses}
%\end{table} % table makes a floating object with a title



% \input{5implementation.tex}


% \input{6evaluation.tex}



%You have done your work, but that's\footnote{By the way, do \emph{not} use







% \input{7discussion.tex}

\chapter{Discussion and conclusions}
\label{chapter:discussionandconclusions}

%At this point, you will have some insightful thoughts on your implementation
%and you may have ideas on what could be done in the future.
%This chapter is a good place to discuss your thesis as a whole and to show your
%professor that you have really understood some non-trivial aspects of the
%methods you used\ldots



% \input{8conclusions.tex}


%Time to wrap it up!
%Write down the most important findings from your work.
%Like the introduction, this chapter is not very long.
%Two to four pages might be a good limit.
    


% Load the bibliographic references
% ------------------------------------------------------------------
% You can use several .bib files:
% \bibliography{thesis_sources,ietf_sources}
\bibliography{ref}


% Appendices go here
% ------------------------------------------------------------------
% If you do not have appendices, comment out the following lines

% \input{appendices.tex}
%\chapter{SUS form}
%\label{chapter:susform}

\appendix
\includepdf[scale=0.9,pages=4,angle=0.0,picturecommand*={%
     \put(110,770){%
         \parbox{\textwidth}{\chapter{SUS form}\label{app:susform}}
     }}]{./pdfs/sus.pdf}



%\includepdf[scale=0.9,pages=4,picturecommand*={%
%     \put(110,770){%
%         \parbox{\textwidth}{\chapter{TEST form}\label{app:susform}}
%     }}]{sus.pdf}


%\includegraphics[scale=0.5,pages={4}]{sus.pdf}

%\includepdf[ pages=4, scale=.7, frame, pagecommand =  \chapter{}]{sus.pdf}
%\chapter{}
%\includegraphics[page=4,width=1.2\textwidth]{sus.pdf}
%\chapter{SUS form}
%\label{chapter:susform}
%\begin{figure}[h]
 %  \centering
   %\begin{tabular}{@{}c@{\vspace{2.5cm}}c@{}}
      % \includegraphics[page=4,width=1.2\textwidth]{sus.pdf} & 

   %\end{tabular}
 %\label{fig:Test}
%\end{figure}



%    \section{APPENDIX A}
%    \label{sec:appendixa}
%This is the first appendix. You could put some test images or verbose data in an
%appendix, if there is too much data to fit in the actual text nicely.

%For now, the Aalto logo variants are shown in Figure~\ref{fig:aaltologo}.

%\begin{figure}
%\begin{center}
%\subfigure[In English]{\includegraphics[width=.8\textwidth]{aalto-logo-en}}
%\subfigure[Suomeksi]{\includegraphics[width=.8\textwidth]{aalto-logo-fi}}
%\subfigure[Pä svenska]{\includegraphics[width=.8\textwidth]{aalto-logo-se}}
%\caption{Aalto logo variants}
%\label{fig:aaltologo}
%\end{center}
%\end{figure}


% End of document!
% ------------------------------------------------------------------
% The LastPage package automatically places a label on the last page.
% That works better than placing a label here manually, because the
% label might not go to the actual last page, if LaTeX needs to place
% floats (that is, figures, tables, and such) to the end of the
% document.
\end{document}
